\documentclass[a4paper,11pt,twoside]{book}
\usepackage{latexsym}
\usepackage{braket}
\usepackage{graphicx}
\usepackage{amsfonts}
\usepackage{amsmath}
\usepackage{verbatim}
\usepackage{amssymb}
\usepackage{amsthm}
\usepackage[english]{babel}
\usepackage[utf8]{inputenc}
\usepackage{listings}
\usepackage{color}
\usepackage{float}
\usepackage{geometry}
\usepackage{algorithm}
\usepackage{algpseudocode}
\usepackage{pdfpages}

\pagestyle{plain}


\definecolor{blue}{RGB}{0,0,180}
\definecolor{darkblue}{RGB}{72,61,139} 
\definecolor{green}{rgb}{0,0.4,0}
\definecolor{gray}{rgb}{0.5,0.5,0.5}
\definecolor{darkgray}{rgb}{0.4,0.4,0.4}
\definecolor{mauve}{rgb}{0.58,0,0.82}
\definecolor{black}{rgb}{0,0,0}
\definecolor{purple}{RGB}{138,43,226}
\definecolor{ocra}{RGB}{218,165,32}
\definecolor{maroon}{rgb}{0.5,0,0}
\definecolor{darkgreen}{rgb}{0,0.5,0}
\definecolor{indianred}{RGB}{178,34,34}

\lstdefinelanguage{XML}
{
  basicstyle=\ttfamily,
  frame=single,            
  captionpos=b,       
  breaklines=true,       
 numbers=left,                    
  numbersep=5pt,                   
  numberstyle=\tiny\color{gray},                               
  morestring=[s]{"}{"},
  morecomment=[s]{?}{?},
  morecomment=[s]{!--}{--},
  commentstyle=\color{darkgreen},
  moredelim=[s][\color{black}]{>}{<},
  moredelim=[s][\color{green}]{\ }{=},
  stringstyle=\color{ocra},
  identifierstyle=\color{maroon}
}

\lstdefinelanguage{CCC}{ 
  breakatwhitespace=false,         
  breaklines=true,                 
  %captionpos=t,                   
  frame=single,                    
  keepspaces=true,                 
  numbers=left,                    
  numbersep=5pt,                   
  numberstyle=\tiny\color{gray}, 
  rulecolor=\color{black},         
  showtabs=false,                  
  stepnumber=1,                    
  tabsize=2,                       
  title=\lstname,                   
  basicstyle=\ttfamily,
  stringstyle=\color{ocra}\ttfamily,
  morestring=[b]",
  identifierstyle=\color{black}\ttfamily,
  morecomment=[l][\color{maroon}]{pragma\ },
  morecomment=[l][\color{gray}]{//},
  morecomment=[s][\color{gray}]{/*}{*/},
  keywordstyle=[1]\color{blue},
  keywords=[1]{int,float,char, class, void, unsigned, struct, bool},
  keywordstyle=[0]\color{indianred},
  keywords=[0]{include,for,if,while,return,public,break}
}


\renewcommand{\lstlistingname}{Code}
\geometry{left=2cm}
\geometry{foot=1cm}
\geometry{top=1cm}
\geometry{right=2cm}
\linespread{1}
\author{Giacomo Dabisias, Filippo Brizzi}
\title{A framework for static allocation of parallel OpenMp code on multi-core platforms}
\frenchspacing


\begin{document}
\tableofcontents
\chapter{C++}
\section{Include files}

\begin{lstlisting}[language=CCC, caption=driver/compiler.h]
#include "llvm/Support/Host.h"
#include "llvm/ADT/IntrusiveRefCntPtr.h"

#include "clang/Frontend/CompilerInstance.h"
#include "clang/Basic/TargetOptions.h"
#include "clang/Basic/TargetInfo.h"
#include "clang/Basic/FileManager.h"
#include "clang/Basic/SourceManager.h"
#include "clang/Lex/Preprocessor.h"
#include "clang/Lex/Lexer.h"
#include "clang/Basic/Diagnostic.h"
#include "clang/AST/ASTContext.h"

/*
 * ---- Custom class to instantiate an object of clang::CompilerInstance with the options and the file
 * 		passed with argv.
 */
class ClangCompiler {

private:
	clang::CompilerInstance compiler_;

public:
	ClangCompiler(int argc, char **argv);
	
	clang::SourceManager &getSourceManager() { return compiler_.getSourceManager(); }
	clang::DiagnosticConsumer getDiagnosticClient() { return compiler_.getDiagnosticClient(); }
	clang::LangOptions getLangOpts() { return compiler_.getLangOpts(); }
	clang::Preprocessor &getPreprocessor() { return compiler_.getPreprocessor(); }
	clang::ASTContext &getASTContext() { return compiler_.getASTContext(); }
	clang::FileManager &getFileManager() { return compiler_.getFileManager(); }

};
\end{lstlisting}

\begin{lstlisting}[language=CCC, caption=driver/program.h]
#include "driver/compiler.h"
#include "utils/source_locations.h"
#include "pragma_handler/Root.h"

#include "clang/StaticAnalyzer/Core/PathSensitive/CheckerContext.h"
#include "clang/Basic/DiagnosticOptions.h"
#include "clang/Frontend/TextDiagnosticPrinter.h"
#include "clang/AST/ASTConsumer.h"
#include "clang/Parse/Parser.h"
#include "clang/Parse/ParseAST.h"
#include "clang/Rewrite/Core/Rewriter.h"
#include "llvm/Support/raw_ostream.h"
#include <string>
#include <iostream>

/*
 * ---- Instantiate a compiler object and start the parser. 
 */
class Program {
/* Contains the list of all the pragmas in the source code */
std::vector<clang::OMPExecutableDirective *> *pragma_list_;
/* Contains the list of all the functions defined in the source code (for profiling pourpuse) */
std::vector<clang::FunctionDecl *> *function_list_;

/* To create the profiling code and the list of pragmas */
void ParseSourceCode(std::string fileName);
/* To create the final source code to be used with the scheduler */
void ParseSourceCode(std::string fileName, std::vector<Root *> *root_vect);

public:
  /* To create the profiling code and the list of pragmas */
	Program(int argc, char **argv) : ccompiler_(argc, argv), pragma_list_(NULL), function_list_(NULL) {
		ParseSourceCode(argv[argc - 1]);
	}

  /* To create the final source code to be used with the scheduler */
  Program(int argc,char **argv, std::vector<Root *> *root_vect) : ccompiler_(argc, argv), pragma_list_(NULL), function_list_(NULL) {
    ParseSourceCode(argv[argc - 1], root_vect);
  }

	std::vector<clang::OMPExecutableDirective *> *getPragmaList() { return pragma_list_; }
	std::vector<clang::FunctionDecl *> *getFunctionList() { return function_list_; }

	ClangCompiler ccompiler_;
};


/*
 * ---- Recursively visit the AST of the source code to exctract the pragmas and rewrite it
 *      adding profile call.
 */
class ProfilingRecursiveASTVisitor: public clang::RecursiveASTVisitor<ProfilingRecursiveASTVisitor> {

  /* Class to rewrite the code */
  clang::Rewriter &rewrite_profiling_;
  
  const clang::SourceManager& sm;

  bool include_inserted_;
  clang::Stmt *previous_stmt_;

  /* Add profiling call to a pragma stmt */
  void RewriteProfiling(clang::Stmt *s);
  /* Given a ForStmt retrieve the value of the condition variable, to know how many cycles will
     do the for */
  std::string ForConditionVarValue(const clang::Stmt *s);
  /* For a given stmt retrive the line of the function where it is defined */
  unsigned GetFunctionLineForPragma(clang::SourceLocation sl);

public:
  ProfilingRecursiveASTVisitor(clang::Rewriter &r_profiling, const clang::SourceManager& sm) : 
          rewrite_profiling_(r_profiling), sm(sm), include_inserted_(false), previous_stmt_(NULL) { }
  
  /* This function is called for each stmt in the AST */
  bool VisitStmt(clang::Stmt *s);
  /* This function is called for each function in the AST */
  bool VisitFunctionDecl(clang::FunctionDecl *f);
  bool VisitDecl(clang::Decl *decl);
  std::vector<clang::OMPExecutableDirective *> pragma_list_;
  std::vector<clang::FunctionDecl *> function_list_;
    
};

/*
 * ---- Is responible to call ProfilingRecurseASTVisitor.
 */
class ProfilingASTConsumer : public clang::ASTConsumer { 
public:

  ProfilingASTConsumer(clang::Rewriter &r_profiling, const clang::SourceManager& sm) : 
          recursive_visitor_(r_profiling, sm) { }
  
  /* Traverse the AST invoking the RecursiveASTVisitor functions */
  virtual bool HandleTopLevelDecl(clang::DeclGroupRef d) {
    typedef clang::DeclGroupRef::iterator iter;
    for (iter b = d.begin(), e = d.end(); b != e; ++b) {
      recursive_visitor_.TraverseDecl(*b);
    } 
    return true; 
  }

  ProfilingRecursiveASTVisitor recursive_visitor_;
  std::vector<clang::OMPExecutableDirective *> pragma_list_;
  std::vector<clang::FunctionDecl *> function_list_;
};


/*
 * ---- Recursively visit the AST and repleace each pragma with a function call.
 */
class TransformRecursiveASTVisitor: public clang::RecursiveASTVisitor<TransformRecursiveASTVisitor> {

  clang::Rewriter &rewrite_pragma_;
  
  const clang::SourceManager& sm;

  /* Needed because the parse retrive twice each pragma stmt */
  clang::Stmt *previous_stmt_;
  /* Check if the inlude command has been already inserted*/
  bool include_inserted_;

  std::vector<Root *> *root_vect_;

  void RewriteOMPPragma(clang::Stmt *associated_stmt, std::string pragma_name);
  void RewriteOMPBarrier(clang::OMPExecutableDirective *omp_stmt);
  std::string RewriteOMPFor(Node *n);
  
  /* Given a pragma stmt retrive the Node object that contains all its info */
  Node *GetNodeObjForPragma(clang::Stmt *s);
  /* Called by GetNodeObjForPragma is used because the Node objs are saved in a tree */
  Node *RecursiveGetNodeObjforPragma(Node *n, unsigned stmt_start_line);


public:
  TransformRecursiveASTVisitor(clang::Rewriter &r_pragma_, std::vector<Root *> *root_vect, const clang::SourceManager& sm) : 
          rewrite_pragma_(r_pragma_), root_vect_(root_vect), sm(sm), include_inserted_(false), previous_stmt_(NULL) { }
  
  bool VisitStmt(clang::Stmt *s);
  bool VisitFunctionDecl(clang::FunctionDecl *f);
  bool VisitDecl(clang::Decl *decl);
};

/*
 * ---- Responsible to invoke TransformRecursiveASTVisitor.
 */
class TransformASTConsumer : public clang::ASTConsumer { 
public:

  TransformASTConsumer(clang::Rewriter &RPragma, std::vector<Root *> *rootVect, const clang::SourceManager& sm) : 
          recursive_visitor_(RPragma, rootVect, sm) { }
  
  virtual bool HandleTopLevelDecl(clang::DeclGroupRef d) {
    typedef clang::DeclGroupRef::iterator iter;
    for (iter b = d.begin(), e = d.end(); b != e; ++b) {
      recursive_visitor_.TraverseDecl(*b);
    } 
    return true; 
  }

  TransformRecursiveASTVisitor recursive_visitor_;
};
\end{lstlisting}

\begin{lstlisting}[language=CCC, caption=pragma\_handler/Node.h]
#include "pragma_handler/ForNode.h"

/* Contains info about function */
struct FunctionInfo {
	clang::FunctionDecl *function_decl_;

	unsigned function_start_line_;
	unsigned function_end_line_;
	std::string function_name_;
	std::string function_return_type_;
	int num_params_;
	/* Matrix Nx2. Contains the list of the parameter of the functions: type name */
	std::string **function_parameters_;

	std::string function_class_name_;
};

/*
 * ---- Contains all the relevant information of a given pragma.
 */
class Node {

private:

	clang::OMPExecutableDirective *pragma_stmt_;

/* Stmt start and end line in the source file */
	std::string file_name_;
	int start_line_, start_column_;
	int end_line_, end_column_;

/*Line number of the function that contains this pragma */
	FunctionInfo parent_funct_info_;

/* Variables to construct the tree */
	Node *parent_node_;

	/*Pragma name with all the parameters */
	//std::string pragma_type_;

/* Function to exctract all the parameters of the pragma */
	void setPragmaClauses(clang::SourceManager& sm);

public:
	/*Pragma name with all the parameters */
	std::string pragma_type_;

	bool profiled_ = false;

	ForNode *for_node_;

	std::vector<Node *> *children_vect_;

	typedef std::map<std::string, std::string> VarList_;
	std::map<std::string, VarList_> *option_vect_;

	Node(clang::OMPExecutableDirective *pragma_stmt, clang::FunctionDecl *funct_decl, clang::SourceManager& sm);

	void setSourceLocation(const clang::SourceManager& sm);

/*
 * ---- Set the line, name, return type and parameters of the function containig the pragma ----
 */
	void setParentFunction(clang::FunctionDecl *funct_decl, const clang::SourceManager& sm);

	FunctionInfo getParentFunctionInfo() { return parent_funct_info_; }

	void AddChildNode(Node *n) { children_vect_->push_back(n); }

	void setParentNode(Node *n) { parent_node_ = n; }
	Node* getParentNode() { return parent_node_; }

	int getEndLine() { return end_line_; }
	int getStartLine() { return start_line_; }

	void CreateXMLPragmaNode(tinyxml2::XMLDocument *xml_doc, tinyxml2::XMLElement *pragmas_element);
	void CreateXMLPragmaOptions(tinyxml2::XMLDocument *xml_doc,tinyxml2:: XMLElement *options_element);
};
\end{lstlisting}

\begin{lstlisting}[language=CCC, caption=pragma\_handler/ForNode.h]
#include "xml_creator/tinyxml2.h"

#include "utils/source_locations.h"
#include "clang/AST/ASTConsumer.h"
#include "clang/Sema/Lookup.h"
#include "clang/Frontend/CompilerInvocation.h"
#include "clang/AST/ASTContext.h"
#include "clang/Sema/Scope.h"
#include "clang/Parse/ParseAST.h"

#include <iostream>
#include <string>
#include <stdio.h>
#include <stdlib.h>

class ForNode {

public:
	clang::ForStmt *for_stmt_;

	/* Loop variable */
	std::string loop_var_;
	std::string loop_var_type_;

	/* Loop variable initial value: (number or variable) */
	int loop_var_init_val_;
	bool loop_var_init_val_set_;
	std::string loop_var_init_var_;

	/* Loop condition */
	std::string condition_op_;
	int condition_val_;
	bool condition_val_set_;
	std::string condition_var_;

	/* Loop increment */
	std::string increment_op_;
	int increment_val_;
	bool increment_val_set_;
	std::string increment_var_;

	void ExtractForParameters(clang::ForStmt *for_stmt);

	void ExtractForInitialization(clang::ForStmt *for_stmt);
	void ExtractForCondition(clang::ForStmt *for_stmt);
	void ExtractForIncrement(clang::ForStmt *for_stmt);


	ForNode(clang::ForStmt *for_stmt);
	void CreateXMLPragmaFor(tinyxml2::XMLDocument *xml_doc, tinyxml2::XMLElement *for_element);

};
\end{lstlisting}

\begin{lstlisting}[language=CCC, caption=pragma\_handler/Root.h]
#include "pragma_handler/Node.h"

/*
 * ---- It's the root node of the annidation tree of the pragmas in a specific function 
 * 		and contains the first level pragmas.
 */
class Root {
private:	
	FunctionInfo function_info_;

	Node *last_node_;

public:
	Root(Node *n, FunctionInfo funct_info);

	std::vector<Node *> *children_vect_;

	void setLastNode(Node *n) {last_node_ = n; };
	Node* getLastNode() { return last_node_; };

	void AddChildNode(Node *n) { children_vect_->push_back(n); };

	void CreateXMLFunction(tinyxml2::XMLDocument *xml_doc);

	unsigned getFunctionLineStart(){ return function_info_.function_start_line_; }
	unsigned getFunctionLineEnd() {return function_info_.function_end_line_; }

};
\end{lstlisting}

\begin{lstlisting}[language=CCC, caption=utils/source\_locations.h]
#include <string>
#include <clang/Basic/SourceLocation.h>
#include <clang/Basic/SourceManager.h>
#include <sstream>

#include <llvm/Support/raw_ostream.h>

namespace clang {
class SourceLocation;
class SourceRange;
class SourceManager;
}

namespace utils {

std::string FileName(clang::SourceLocation const& l, clang::SourceManager const& sm);

std::string FileId(clang::SourceLocation const& l, clang::SourceManager const& sm);

unsigned Line(clang::SourceLocation const& l, clang::SourceManager const& sm);

std::pair<unsigned, unsigned> Line(clang::SourceRange const& r, clang::SourceManager const& sm);

unsigned Column(clang::SourceLocation const& l, clang::SourceManager const& sm);

std::pair<unsigned, unsigned> Column(clang::SourceRange const& r, clang::SourceManager const& sm);

std::string location(clang::SourceLocation const& l, clang::SourceManager const& sm);

}
\end{lstlisting}

\section{Source files}

\begin{lstlisting}[language=CCC, caption=main.cpp]
int main(int argc, char **argv) { 

	if(argc < 2) {
		llvm::errs() << "Usage: Source_exctractor [<options>] <filename>\n";
     	return 1;
	}
/*
 * ---- Create a clang::compiler object and launch the parser saving the pragma stmt. 
 * 		Rewrite the sourcecode adding profiling call.
 */
	Program p_parser(argc, argv);

/*
 * ---- With the information exctracted by the parser create a linked list tree of objects containing
 *		all the necessary information of the pragmas.
 */
	std::vector<Root *> *root_vect = CreateTree(program.getPragmaList(), program.getFunctionList(), program.ccompiler_.getSourceManager());
/*
 * ---- Using the tree above create an xml file containing the pragma info. This file is used to produce the scheduler.
 */
	CreateXML(root_vect, argv[argc - 1]);

	for(std::vector<Root *>::iterator itr = root_vect->begin(); itr != root_vect->end(); ++itr) 
		(*itr)->VisitTree();

/*
 * ---- Parse the sourcecode and rewrite it substituting pragmas with function calls. This new file
 * 		will be used with the scheduler to produce the final output.
 */
	Program p_rewriter(argc, argv, root_vect);

	return 0;
}
\end{lstlisting}

\begin{lstlisting}[language=CCC, caption=driver/compiler.cpp]
#include "driver/compiler.h"

using namespace clang;

ClangCompiler::ClangCompiler(int argc, char **argv) {

  DiagnosticOptions diagnosticOptions;
  compiler_.createDiagnostics();

  /* Create an invocation that passes any flags to preprocessor */
  CompilerInvocation *Invocation = new CompilerInvocation;
  CompilerInvocation::CreateFromArgs(*Invocation, argv + 1, argv + argc,
                                      compiler_.getDiagnostics());
  compiler_.setInvocation(Invocation);

  /* Set default target triple */
  llvm::IntrusiveRefCntPtr<TargetOptions> pto( new TargetOptions());
  pto->Triple = llvm::sys::getDefaultTargetTriple();
  llvm::IntrusiveRefCntPtr<TargetInfo> pti(TargetInfo::CreateTargetInfo(compiler_.getDiagnostics(), pto.getPtr()));
  compiler_.setTarget(pti.getPtr());

  compiler_.createFileManager();
  compiler_.createSourceManager(compiler_.getFileManager());

  /* Add default search path for the compiler */
  HeaderSearchOptions &headerSearchOptions = compiler_.getHeaderSearchOpts();

  headerSearchOptions.AddPath("/usr/local/include",
	          clang::frontend::Angled,
	          false,
	          false);

	headerSearchOptions.AddPath("/usr/include",
	          clang::frontend::Angled,
	          false,
	          false);

	headerSearchOptions.AddPath("/usr/lib/gcc/x86_64-linux-gnu/4.8/include",
	          clang::frontend::Angled,
	          false,
	          false);

	headerSearchOptions.AddPath("/usr/include/x86_64-linux-gnu",
	          clang::frontend::Angled,
	          false,
	          false);
  headerSearchOptions.AddPath("/usr/include/c++/4.8/",
            clang::frontend::Angled,
            false,
            false);

  headerSearchOptions.AddPath("/usr/include/x86_64-linux-gnu/c++/4.8/",
            clang::frontend::Angled,
            false,
            false);


  /* Allow C++ code to get rewritten */
  clang::LangOptions langOpts;
  langOpts.GNUMode = 1; 
  langOpts.CXXExceptions = 1; 
  langOpts.RTTI = 1; 
  langOpts.Bool = 1; 
  langOpts.CPlusPlus = 1; 
  Invocation->setLangDefaults(langOpts,
                              clang::IK_CXX,
                              clang::LangStandard::lang_cxx0x);

  compiler_.createPreprocessor();
  compiler_.getPreprocessorOpts().UsePredefines = false;

  compiler_.createASTContext();

  /* Initialize the compiler and the source manager with a file to process */
  std::string fileName(argv[argc - 1]);  
  const FileEntry *pFile = compiler_.getFileManager().getFile(fileName);
  compiler_.getSourceManager().createMainFileID(pFile);
  compiler_.getDiagnosticClient().BeginSourceFile(compiler_.getLangOpts(), &compiler_.getPreprocessor());

}
\end{lstlisting}

\begin{lstlisting}[language=CCC, caption=driver/program.cpp]
#include "driver/program.h"

void Program::ParseSourceCode(std::string file_name) {

  /* Convert <file>.c to <file_profile>.c */
  std::string out_filename_profile (file_name);
  size_t ext = out_filename_profile.rfind(".");
  if (ext == std::string::npos)
  	ext = out_filename_profile.length();
  out_filename_profile.insert(ext, "_profile");

  llvm::errs() << "Output to: " << out_filename_profile << "\n";
  std::string out_error_info;
  llvm::raw_fd_ostream out_file_profile(out_filename_profile.c_str(), out_error_info, 0);

  /* Create the rewriter object to create the profiling file */
  clang::Rewriter rewrite_profiling;
  rewrite_profiling.setSourceMgr(ccompiler_.getSourceManager(), ccompiler_.getLangOpts());
  
  ProfilingASTConsumer ast_consumer(rewrite_profiling, ccompiler_.getSourceManager());
  /* Parse the AST with the custom ASTConsumer */
  clang::ParseAST(ccompiler_.getPreprocessor(), &ast_consumer, ccompiler_.getASTContext());
	ccompiler_.getDiagnosticClient().EndSourceFile();

  /* Save the pragma and function list */
	pragma_list_ = new std::vector<clang::OMPExecutableDirective *>(ast_consumer.recursive_visitor_.pragma_list_);
	function_list_ = new std::vector<clang::FunctionDecl *>(ast_consumer.recursive_visitor_.function_list_);

  /*Output rewritten source code into a new file */
	const clang::RewriteBuffer *rewrite_buf_profiling = 
      rewrite_profiling.getRewriteBufferFor(ccompiler_.getSourceManager().getMainFileID());  	
  
  out_file_profile << std::string(rewrite_buf_profiling->begin(), rewrite_buf_profiling->end());
  out_file_profile.close();

}

bool ProfilingRecursiveASTVisitor::VisitDecl(clang::Decl *decl) {

  clang::SourceLocation cxx_start_src_loc = decl->getLocStart();
  if(sm.getFileID(cxx_start_src_loc) == sm.getMainFileID() 
      && clang::isa<clang::CXXRecordDecl>(decl)
      && include_inserted_ == false) {
    include_inserted_ = true;
    std::string text_include = 
      "#include \"profile_tracker/profile_tracker.h\"\n";
    rewrite_profiling_.InsertText(cxx_start_src_loc, text_include, true, false);
  }

  return true;
}

/*
 * ---- Insert the call to the profilefunction tracker to track the execution time of each funcion.
 */
bool ProfilingRecursiveASTVisitor::VisitFunctionDecl(clang::FunctionDecl *f) {     
    
  clang::SourceLocation start_src_loc = f->getLocStart();
  unsigned funct_start_line = utils::Line(start_src_loc, sm);


  /* Skip function belonging to external include file and not defined function */
  if(sm.getFileID(start_src_loc) == sm.getMainFileID() && f->hasBody() == true) {
  	
    function_list_.push_back(f);

    /* Include the path to ProfileTracker.h */
    if(include_inserted_ == false) {
      std::string text_include = 
      "#include \"profile_tracker/profile_tracker.h\"\n";
      
      rewrite_profiling_.InsertText(start_src_loc, text_include, true, false);
      include_inserted_ = true;
    } 

    start_src_loc = f->getBody()->getLocStart();
  	unsigned start_line = utils::Line(start_src_loc, sm);
    clang::SourceLocation new_start_src_loc = sm.translateLineCol(sm.getMainFileID(), start_line + 1, 1);
    std::stringstream text_profiling;
    text_profiling << "if( ProfileTracker x = ProfileTrackParams(" << funct_start_line << ", 0)) {\n";
    
    /* Insert the if in the first line of the function definition */
    rewrite_profiling_.InsertText(new_start_src_loc, text_profiling.str(), true, false);

    clang::SourceLocation end_src_loc = f->getLocEnd();
    std::stringstream text_end_bracket;
    text_end_bracket << "}\n";
    /* Close the if bracket at the end of the function */
    rewrite_profiling_.InsertText(end_src_loc, text_end_bracket.str(), true, false);
  }

  return true; 
}

bool ProfilingRecursiveASTVisitor::VisitStmt(clang::Stmt *s) {

	clang::SourceLocation start_src_loc = s->getLocStart();
	if(sm.getFileID(start_src_loc) == sm.getMainFileID()) {
      /* We want just the OpenMP stmt and no duplicate */
  		if (clang::isa<clang::OMPExecutableDirective>(s) && s != previous_stmt_) {
  			previous_stmt_ = s;
  			clang::OMPExecutableDirective *omp_stmt = static_cast<clang::OMPExecutableDirective *>(s);
  			pragma_list_.push_back(omp_stmt);
  			
        clang::Stmt *associated_stmt = omp_stmt->getAssociatedStmt();
        if(associated_stmt) {
          clang::Stmt *captured_stmt = static_cast<clang::CapturedStmt *>(associated_stmt)->getCapturedStmt();
          /* In the case of #omp parallel for we have to go down two level befor finding the ForStmt */
  		    if(strcmp(captured_stmt->getStmtClassName(), "OMPForDirective") != 0)
            RewriteProfiling(captured_stmt);
        }
      }
  	}
  	
  return true;
}

void ProfilingRecursiveASTVisitor::RewriteProfiling(clang::Stmt *s) {
  	
  	clang::SourceLocation start_src_loc = s->getLocStart();
  	unsigned pragma_start_line = utils::Line(start_src_loc,sm);
  	unsigned function_start_line = GetFunctionLineForPragma(s->getLocStart());

    std::stringstream text_profiling;
    if(clang::isa<clang::ForStmt>(s)) {
      std::string condition_var_value = ForConditionVarValue(s);
      //std::string conditionVar = "";
      text_profiling << "if( ProfileTracker x = ProfileTrackParams(" 
          << function_start_line << ", " << pragma_start_line << ", " << condition_var_value << "))\n";
      rewrite_profiling_.InsertText(start_src_loc, text_profiling.str(), true, true);

    } else {
	    text_profiling << "if( ProfileTracker x = ProfileTrackParams(" 
          << function_start_line << ", " << pragma_start_line << "))\n";
      rewrite_profiling_.InsertText(start_src_loc, text_profiling.str(), true, true);
    }

    /* Comment the pragma in the profiling file */
    clang::SourceLocation pragma_start_src_loc = 
        sm.translateLineCol(sm.getMainFileID(), pragma_start_line - 1, 1);
    
    rewrite_profiling_.InsertText(pragma_start_src_loc, "//", true, false);
}

std::string ProfilingRecursiveASTVisitor::ForConditionVarValue(const clang::Stmt *s) {

  const clang::ForStmt *for_stmt = static_cast<const clang::ForStmt *>(s);
  const clang::Expr *condition_expr = for_stmt->getCond();
  const clang::BinaryOperator *binary_op = static_cast<const clang::BinaryOperator *>(condition_expr);
  
  std::string start_cond_var_value, end_cond_var_value;

/*
 *  Condition end value
 */
  const clang::Expr *right_expr = binary_op->getRHS();

  if(strcmp(right_expr->getStmtClassName(), "IntegerLiteral") == 0) {
    const clang::IntegerLiteral *int_literal = static_cast<const clang::IntegerLiteral *>(right_expr);
    std::stringstream text_end_value;
    text_end_value << int_literal->getValue().getZExtValue();
    //return text.str();
    end_cond_var_value = text_end_value.str();

  } else if(strcmp(right_expr->getStmtClassName(), "ImplicitCastExpr") == 0) {
  	const clang::DeclRefExpr *decl_ref_expr = 
        static_cast<const clang::DeclRefExpr *>(*(right_expr->child_begin()));
  	
    const clang::NamedDecl *named_decl = decl_ref_expr->getFoundDecl();
  	//return nD->getNameAsString(); 
    end_cond_var_value = named_decl->getNameAsString();
  }

/*
 * Condition start value
 */

/*
 *  for (int i = ...)
 */
  if(strcmp(for_stmt->child_begin()->getStmtClassName(), "DeclStmt") == 0) {
    const clang::DeclStmt *decl_stmt = static_cast<const clang::DeclStmt *>(*(for_stmt->child_begin()));
    const clang::Decl *decl = decl_stmt->getSingleDecl();

/*
 *  for (... = 0)
 */
    if(strcmp(decl_stmt->child_begin()->getStmtClassName(), "IntegerLiteral") == 0) {      
      const clang::IntegerLiteral *int_literal = 
          static_cast<const clang::IntegerLiteral *>(*(decl_stmt->child_begin())); 
      
      std::stringstream text_star_value;
      text_star_value << int_literal->getValue().getZExtValue();
      start_cond_var_value = text_star_value.str();

/*
 *  for (... = a)
 */
    }else if (strcmp(decl_stmt->child_begin()->getStmtClassName(), "ImplicitCastExpr") == 0) {
      const clang::DeclRefExpr *decl_ref_expr = 
          static_cast<const clang::DeclRefExpr *>(*(decl_stmt->child_begin()->child_begin()));
      
      const clang::NamedDecl *named_decl = decl_ref_expr->getFoundDecl();
      start_cond_var_value = named_decl->getNameAsString();
    }
  }
/*
 *  for ( i = ...)
 */
  else if(strcmp(for_stmt->child_begin()->getStmtClassName(), "BinaryOperator") == 0) {
    const clang::BinaryOperator *binary_op = 
        static_cast<const clang::BinaryOperator *>(*(for_stmt->child_begin())); 
    const clang::DeclRefExpr *decl_ref_expr = 
        static_cast<const clang::DeclRefExpr *>(*(binary_op->child_begin()));
/*
 *  for( ... = 0)
 */
    clang::ConstStmtIterator stmt_itr = binary_op->child_begin();
    stmt_itr ++;
    if(strcmp(stmt_itr->getStmtClassName(), "IntegerLiteral") == 0) {
      const clang::IntegerLiteral *int_literal = static_cast<const clang::IntegerLiteral *>(*stmt_itr);
      start_cond_var_value = int_literal->getValue().getZExtValue();      
/*
 *  for ( ... = a)
 */
    } else if (strcmp(stmt_itr->getStmtClassName(), "ImplicitCastExpr") == 0) {
      const clang::DeclRefExpr *decl_ref_expr = 
          static_cast<const clang::DeclRefExpr *>(*(stmt_itr->child_begin()));
      const clang::NamedDecl *named_decl = decl_ref_expr->getFoundDecl();
      start_cond_var_value = named_decl->getNameAsString();
    }
  }
  end_cond_var_value.append(" - ");
  end_cond_var_value.append(start_cond_var_value);
  return end_cond_var_value;
}

unsigned ProfilingRecursiveASTVisitor::GetFunctionLineForPragma(clang::SourceLocation sl) {

	unsigned pragma_line = utils::Line(sl, sm);

  unsigned start_func_line, end_func_line;
	std::vector<clang::FunctionDecl *>::iterator func_itr;

	for(func_itr = function_list_.begin(); func_itr != function_list_.end(); ++ func_itr) {
		start_func_line = utils::Line((*func_itr)->getSourceRange().getBegin(), sm);
		end_func_line = utils::Line((*func_itr)->getSourceRange().getEnd(), sm);
		if(pragma_line < end_func_line && pragma_line > start_func_line)
			return start_func_line;
	}

	return 0;
}


void Program::ParseSourceCode(std::string fileName, std::vector<Root *> *root_vect) {

  /* Convert <file>.c to <file_transformed>.c */
  std::string out_name_pragma (fileName);
  size_t ext = out_name_pragma.rfind(".");
  if (ext == std::string::npos)
    ext = out_name_pragma.length();
  out_name_pragma.insert(ext, "_transformed");

  llvm::errs() << "Output to: " << out_name_pragma << "\n";
  std::string out_error_info;
  llvm::raw_fd_ostream out_file_pragma(out_name_pragma.c_str(), out_error_info, 0);  

  clang::Rewriter rewrite_pragma;
  rewrite_pragma.setSourceMgr(ccompiler_.getSourceManager(), ccompiler_.getLangOpts());

  TransformASTConsumer t_ast_consumer(rewrite_pragma, root_vect, ccompiler_.getSourceManager());
  
  /* Parse the AST */
  clang::ParseAST(ccompiler_.getPreprocessor(), &t_ast_consumer, ccompiler_.getASTContext());
  ccompiler_.getDiagnosticClient().EndSourceFile();

  const clang::RewriteBuffer *rewrite_buff_pragma = 
      rewrite_pragma.getRewriteBufferFor(ccompiler_.getSourceManager().getMainFileID());
  out_file_pragma << std::string(rewrite_buff_pragma->begin(), rewrite_buff_pragma->end());
  out_file_pragma.close();
}

bool TransformRecursiveASTVisitor::VisitDecl(clang::Decl *decl) {

  clang::SourceLocation cxx_start_src_loc = decl->getLocStart();
  if(sm.getFileID(cxx_start_src_loc) == sm.getMainFileID() 
      && clang::isa<clang::CXXRecordDecl>(decl)
      && include_inserted_ == false) {
    include_inserted_ = true;
    std::string text_include = "#include \"thread_pool/threads_pool.h\"\n";
    rewrite_pragma_.InsertText(cxx_start_src_loc, text_include, true, false);
  }

  return true;
}

bool TransformRecursiveASTVisitor::VisitFunctionDecl(clang::FunctionDecl *f) {     
  clang::SourceLocation f_start_src_loc = f->getLocStart();

  if(sm.getFileID(f_start_src_loc) == sm.getMainFileID() && !clang::isa<clang::CXXMethodDecl>(f)) {
    if(include_inserted_ == false) {
      include_inserted_ = true;

      std::string text_include = "#include \"thread_pool/threads_pool.h\"\n";

      rewrite_pragma_.InsertText(f_start_src_loc, text_include, true, false);
    }
  }

  return true;
}

bool TransformRecursiveASTVisitor::VisitStmt(clang::Stmt *s) {
  
  clang::SourceLocation s_start_stc_loc = s->getLocStart();
  /* Visit only stmt in the source file (not in included file) and that are pragma stmt */
  if(sm.getFileID(s_start_stc_loc) == sm.getMainFileID() 
        && clang::isa<clang::OMPExecutableDirective>(s) 
        && s != previous_stmt_) {
    
    previous_stmt_ = s;
    clang::OMPExecutableDirective *omp_stmt = static_cast<clang::OMPExecutableDirective *>(s);
    clang::Stmt *associated_stmt = omp_stmt->getAssociatedStmt();
    if(associated_stmt) {
      clang::Stmt *captured_stmt = static_cast<clang::CapturedStmt *>(associated_stmt)->getCapturedStmt();
      if(strcmp(captured_stmt->getStmtClassName(), "OMPForDirective") != 0)
        RewriteOMPPragma(associated_stmt, omp_stmt->getStmtClassName());

    }else if(strcmp(omp_stmt->getStmtClassName(), "OMPBarrierDirective") == 0
            || strcmp(omp_stmt->getStmtClassName(), "OMPTaskwaitDirective") == 0){
      RewriteOMPBarrier(omp_stmt);
    }
  }
  return true;
}

void TransformRecursiveASTVisitor::RewriteOMPBarrier(clang::OMPExecutableDirective *omp_stmt) {
  unsigned stmt_start_line = utils::Line(omp_stmt->getLocStart(), sm);
  
  std::stringstream text_barrier;
  text_barrier <<
"{\n\
  class Nested : public NestedBase {\n\
  public: \n\
    virtual std::shared_ptr<NestedBase> clone() const { return std::make_shared<Nested>(*this); } \n\
    Nested(int pragma_id) : NestedBase(pragma_id) {}\n\
    void callme(ForParameter for_param){}\n\
  };\n\
  ThreadPool::getInstance(\"" << utils::FileName(omp_stmt->getLocStart(), sm) 
      << "\")->call(std::make_shared<Nested>(" << stmt_start_line << "));\n\
}";

  clang::SourceLocation pragma_start_src_loc = sm.translateLineCol(sm.getMainFileID(), stmt_start_line + 1, 1);
  rewrite_pragma_.InsertText(pragma_start_src_loc, text_barrier.str(), true, false);

  pragma_start_src_loc = sm.translateLineCol(sm.getMainFileID(), stmt_start_line, 1);
  rewrite_pragma_.InsertText(pragma_start_src_loc, "//", true, false);
}


void TransformRecursiveASTVisitor::RewriteOMPPragma(clang::Stmt *associated_stmt, std::string pragma_name) {
  
  clang::Stmt *s = static_cast<clang::CapturedStmt *>(associated_stmt)->getCapturedStmt();

  clang::SourceLocation stmt_start_src_loc = s->getLocStart();
  unsigned pragma_start_line = utils::Line(stmt_start_src_loc, sm);

  Node *n = GetNodeObjForPragma(s);
  
  std::stringstream text;
  std::stringstream text_constructor_params;
  std::stringstream text_class_var;
  std::stringstream text_fx_var;
  std::stringstream text_constructor_var;
  std::stringstream text_constructor;

/* Insert before pragma */
  text <<
"{\n\
  class Nested : public NestedBase {\n\
  public: \n\
    virtual std::shared_ptr<NestedBase> clone() const { return std::make_shared<Nested>(*this); } \n\
    Nested(int pragma_id";

  text_constructor << " : NestedBase(pragma_id)";

  clang::CapturedStmt *captured_stmt = static_cast<clang::CapturedStmt *>(associated_stmt);
  /* Iterate over all the variable used inside a pragma but defined outside. These variable have to be passed to
     the newly created function */ 
  for(clang::CapturedStmt::capture_iterator capture_var_itr = captured_stmt->capture_begin(); 
      capture_var_itr != captured_stmt->capture_end(); 
      ++capture_var_itr){

    clang::VarDecl *var_decl = capture_var_itr->getCapturedVar(); 
    std::string var_type = var_decl->getType().getAsString();

    if(capture_var_itr != captured_stmt->capture_begin()){
      text_fx_var << ", ";
      text_constructor_var << ", ";
      text_constructor_params << ", ";
    }else
      text << ", ";
    std::cout << var_type << " - ";
    size_t pos_class = var_type.find("class");
    if(pos_class != std::string::npos){
      std::cout << "removing class - ";
      var_type.erase(pos_class, pos_class + 5);
    }
    
    size_t pos_uppersand = var_type.find("&");
    if(pos_uppersand != std::string::npos)
      var_type.erase(pos_uppersand - 1, var_type.size());

    if(n->option_vect_->find("private") != n->option_vect_->end()) {
      if(n->option_vect_->find("private")->second.find(var_decl->getNameAsString()) 
            != n->option_vect_->find("private")->second.end() 
            || var_type.find("*") != std::string::npos){

        text_constructor_params << var_type << " " << var_decl->getNameAsString();
        text_class_var << var_type << " " << var_decl->getNameAsString() << "_;\n";

      }else{
        text_constructor_params << var_type << " & " << var_decl->getNameAsString();
        text_class_var << var_type << " & " << var_decl->getNameAsString() << "_;\n";
      }
    }else if(var_type.find("*") != std::string::npos) {
      text_constructor_params << var_type << " " << var_decl->getNameAsString();
      text_class_var << var_type << " " << var_decl->getNameAsString() << "_;\n";

    }else {
      text_constructor_params << var_type << " & " << var_decl->getNameAsString();
      text_class_var << var_type << " & " << var_decl->getNameAsString() << "_;\n";
    }
    std::cout << var_type << std::endl;

    text_constructor << ", " << var_decl->getNameAsString() << "_(" << var_decl->getNameAsString() << ") ";
    text_fx_var << var_decl->getNameAsString() << "_";
    text_constructor_var << var_decl->getNameAsString();
  }

  text << text_constructor_params.str() << ") " << text_constructor.str() 
      << "{}\n" << text_class_var.str() << "\n";    
  
  unsigned stmt_start_line = utils::Line(s->getLocStart(), sm);
  
  if(text_constructor_params.str().compare("") == 0)
      text << "void fx(ForParameter for_param)";
    else
      text << "void fx(ForParameter for_param, " << text_constructor_params.str() <<")";
  
  unsigned stmt_end_line = utils::Line(s->getLocEnd(), sm);
  if(n->for_node_ != NULL) {
    
    std::string text_for;
    text_for = RewriteOMPFor(n);

    text << " {\n" << text_for;
    clang::SourceLocation for_src_loc = sm.translateLineCol(sm.getMainFileID(), stmt_start_line + 1, 1);
    rewrite_pragma_.InsertText(for_src_loc, text.str(), true, false);
    rewrite_pragma_.InsertText(stmt_start_src_loc, "//", true, false);
    
    clang::SourceLocation for_end_src_loc = sm.translateLineCol(sm.getMainFileID(), stmt_end_line + 1, 1);
    rewrite_pragma_.InsertText(for_end_src_loc, "launch_todo_job(); \n }\n", true, false);
  }else {
    rewrite_pragma_.InsertText(stmt_start_src_loc, text.str(), true, true);
    //clang::SourceLocation stmt_end_src_loc = sm.translateLineCol(sm.getMainFileID(), stmt_end_line - 1, 1);
    
    rewrite_pragma_.InsertText(s->getLocEnd(), "launch_todo_job(); \n", true, false);
  }

  /* Comment the pragma */
  clang::SourceLocation pragma_src_loc = sm.translateLineCol(sm.getMainFileID(), stmt_start_line - 1, 1);
  rewrite_pragma_.InsertText(pragma_src_loc, "//", true, false);
  
/*
 * ----- Insert after pragma ----
 */
   
  std::stringstream text_after_pragma;
  text_after_pragma <<"\
void callme(ForParameter for_param) {\n";

  if(text_fx_var.str().compare("") == 0)
    text_after_pragma << "  fx(for_param);\n";
  else
    text_after_pragma << "  fx(for_param, " << text_fx_var.str() << ");\n";

text_after_pragma << 
"}\n\
};\n\
std::shared_ptr<NestedBase> nested_b = std::make_shared<Nested>(" << n->getStartLine();
if(text_constructor_var.str().compare("") != 0)
      text_after_pragma << ", ";    
text_after_pragma << text_constructor_var.str() <<");\n";
text_after_pragma << 
"if(ThreadPool::getInstance(\"" << utils::FileName(s->getLocStart(), sm) << "\")->call(nested_b)) \n";
    
  std::cout << "classname " << pragma_name << std::endl;

  if(pragma_name.compare("OMPParallelDirective") == 0 || pragma_name.compare("OMPForDirective") == 0) {

text_after_pragma << "  nested_b->callme(ForParameter(0,1));\n";
  }else {
text_after_pragma << "  todo_job_.push(nested_b); \n";
  }
text_after_pragma << "}\n";

/* If ForDirective no need to add the if, cause everything is solved inside */
  stmt_end_line = utils::Line(s->getLocEnd(), sm);
  clang::SourceLocation pragma_end_src_loc = sm.translateLineCol(sm.getMainFileID(), stmt_end_line + 1, 1);

  rewrite_pragma_.InsertText(pragma_end_src_loc, text_after_pragma.str(), true, false);

}

Node *TransformRecursiveASTVisitor::GetNodeObjForPragma(clang::Stmt *s){

  clang::SourceLocation stmt_start_src_loc = s->getLocStart();
  unsigned stmt_start_line = utils::Line(stmt_start_src_loc, sm);

  std::vector<Root *>::iterator root_itr;
  for(root_itr = root_vect_->begin(); root_itr != root_vect_->end(); root_itr ++) {
    if((*root_itr)->getFunctionLineStart() < utils::Line(stmt_start_src_loc, sm) 
        && (*root_itr)->getFunctionLineEnd() > utils::Line(stmt_start_src_loc, sm))
      
      break;
  }
  
  std::vector<Node *>::iterator node_itr;
  Node * n;
  for(node_itr = (*root_itr)->children_vect_->begin(); 
      node_itr != (*root_itr)->children_vect_->end(); 
      node_itr ++) {
    
    n = RecursiveGetNodeObjforPragma(*node_itr, stmt_start_line);
    if(n != NULL)
      return n;
  }
  return NULL;
}

Node *TransformRecursiveASTVisitor::RecursiveGetNodeObjforPragma(Node *n, unsigned stmt_start_line) {
  Node *nn;
  if(n->getStartLine() == stmt_start_line){
        return n;
  }else if(n->children_vect_ != NULL) {
    for(std::vector<Node *>::iterator node_itr = n->children_vect_->begin(); 
        node_itr != n->children_vect_->end(); ++ node_itr) {
      
      nn = RecursiveGetNodeObjforPragma(*node_itr, stmt_start_line);
      if(nn != NULL)
        return nn;
    }
  }
  return NULL;
}


std::string TransformRecursiveASTVisitor::RewriteOMPFor(Node *n) {

  std::stringstream text_for;

  ForNode *for_node = n->for_node_;

  /* for( int i = a + for_param->thread_id_ *(b - a)/ num_threads_; .... */
  text_for << "for(" << for_node->loop_var_type_ << " " << for_node->loop_var_ << " = ";
  if(for_node->loop_var_init_val_set_)
    text_for << for_node->loop_var_init_val_;
  else
    text_for << for_node->loop_var_init_var_;

  text_for << " + for_param.thread_id_*(";
  if(for_node->condition_val_set_)
    text_for << for_node->condition_val_ << " - ";
  else
    text_for << for_node->condition_var_ << " - ";

  if(for_node->loop_var_init_val_set_)
    text_for << for_node->loop_var_init_val_;
  else
    text_for << for_node->loop_var_init_var_;

  text_for << ")/for_param.num_threads_; "; 


  /* ....; i < a + (for_param->thread_id_ + 1)*(b - a)/ num_threads_; ... */
  text_for << for_node->loop_var_ << " " << for_node->condition_op_ << " ";

  if(for_node->loop_var_init_val_set_)
    text_for << for_node->loop_var_init_val_;
  else
    text_for << for_node->loop_var_init_var_;

  text_for << " + (for_param.thread_id_ + 1)*(";
  if(for_node->condition_val_set_)
    text_for << for_node->condition_val_ << " - ";
  else
    text_for << for_node->condition_var_ << " - ";

  if(for_node->loop_var_init_val_set_)
    text_for << for_node->loop_var_init_val_;
  else
    text_for << for_node->loop_var_init_var_;

  text_for << ")/for_param.num_threads_; "; 
  

  /* ...; i ++) */
  text_for << for_node->loop_var_ << " " << for_node->increment_op_ << " ";
  if(for_node->increment_val_set_)
    text_for << for_node->increment_val_;
  else
    text_for << for_node->increment_var_; 

  /* Guarantee that a "{" is inserted at the end of the for declaration line if necessary */ 
  clang::SourceLocation for_src_loc = for_node->for_stmt_->getLocStart();
  std::string for_string = sm.getCharacterData(for_src_loc);
  size_t ext = for_string.find_first_of("\n");
  for_string = for_string.substr(0, ext);

  ext = for_string.rfind("{");
  if (ext == std::string::npos)
    text_for << ")\n";
  else
    text_for << ") { \n";

  return text_for.str();

}
\end{lstlisting}

\begin{lstlisting}[language=CCC, caption=pragma\_handler/Node.cpp]
#include "pragma_handler/Node.h"

Node::Node(clang::OMPExecutableDirective *pragma_stmt, clang::FunctionDecl *funct_decl, clang::SourceManager& sm){
  
  option_vect_ = new std::map<std::string, VarList_>();
  pragma_stmt_ = pragma_stmt;

  if(pragma_stmt->getAssociatedStmt()) {
    if(strcmp(pragma_stmt->getStmtClassName(), "OMPParallelDirective") == 0 && utils::Line(pragma_stmt->getAssociatedStmt()->getLocStart(), sm) == utils::Line(pragma_stmt->getAssociatedStmt()->getLocEnd(), sm)){
      setPragmaClauses(sm);
      pragma_stmt_ = static_cast<clang::OMPExecutableDirective *>(static_cast<clang::CapturedStmt *>(pragma_stmt->getAssociatedStmt())->getCapturedStmt());
    }
  }
  setSourceLocation(sm);
	setParentFunction(funct_decl, sm);	
	setPragmaClauses(sm);

	children_vect_ = new std::vector<Node *>();

	if(strcmp(pragma_stmt_->getStmtClassName(), "OMPForDirective") == 0) {
    clang::ForStmt *for_stmt = static_cast<clang::ForStmt *>(static_cast<clang::CapturedStmt *>(pragma_stmt_->getAssociatedStmt())->getCapturedStmt());
    for_node_ = new ForNode(for_stmt);
  } else
		for_node_ = NULL;
}

void Node::setSourceLocation(const clang::SourceManager& sm) {
  
  clang::Stmt *s = pragma_stmt_;
  if(pragma_stmt_->getAssociatedStmt())
    s = static_cast<clang::CapturedStmt *>(pragma_stmt_->getAssociatedStmt())->getCapturedStmt();

  file_name_ = utils::FileName(pragma_stmt_->getLocStart(), sm);
  if(s != NULL) {
    start_line_ = utils::Line(s->getLocStart(), sm);
    start_column_ = utils::Column(s->getLocStart(), sm);

    end_line_ = utils::Line(s->getLocEnd(), sm);
    end_column_ = utils::Column(s->getLocEnd(), sm);

  } else {
    start_line_ = utils::Line(pragma_stmt_->getLocStart(), sm);
    start_column_ = utils::Column(pragma_stmt_->getLocStart(), sm);

    end_line_ = utils::Line(pragma_stmt_->getLocEnd(), sm);
    end_column_ = utils::Column(pragma_stmt_->getLocEnd(), sm);
  }
  return;
}

void Node::setParentFunction(clang::FunctionDecl *funct_decl, const clang::SourceManager& sm) {
  
  parent_funct_info_.function_decl_ = funct_decl;
  parent_funct_info_.function_start_line_ =  utils::Line(funct_decl->getLocStart(), sm);
  parent_funct_info_.function_end_line_ = utils::Line(funct_decl->getLocEnd(), sm);
  
  /* Name of the function containing the pragma */
  parent_funct_info_.function_name_ = funct_decl->getNameInfo().getAsString();

  /* Return type of the function containing the pragma */
  parent_funct_info_.function_return_type_ = funct_decl->getResultType().getAsString();

  /* Parameters of the function containing the pragma */
  parent_funct_info_.num_params_ = funct_decl->getNumParams();
  parent_funct_info_.function_parameters_ = new std::string*[parent_funct_info_.num_params_];
  
  for(int i = 0; i < parent_funct_info_.num_params_; i ++) {
    parent_funct_info_.function_parameters_[i] = new std::string[2];
    
    const clang::ValueDecl *value_decl = static_cast<const clang::ValueDecl *>(funct_decl->getParamDecl(i));
    parent_funct_info_.function_parameters_[i][0] = value_decl->getType().getAsString();

    const clang::NamedDecl *named_decl = static_cast<const clang::NamedDecl *>(funct_decl->getParamDecl(i));            
    parent_funct_info_.function_parameters_[i][1] = named_decl->getNameAsString();
  }

  /* If the parent function is declared in a class return the name of the class */
/*  if (clang::CXXMethodDecl *cxxMethodD = dynamic_cast<clang::CXXMethodDecl *>(funct_decl)){
      const clang::NamedDecl *nD = static_cast<const clang::NamedDecl *>(cxxMethodD->getParent());            
      parent_funct_info_.parentFunctionClassName = nD->getQualifiedNameAsString();
  }
*/
    parent_funct_info_.function_class_name_ = "";
}

void Node::setPragmaClauses(clang::SourceManager& sm) {
  
  pragma_type_ = pragma_stmt_->getStmtClassName();
/*
 * ---- Exctract pragma options ----
 */
  clang::OMPClause *omp_clause = NULL;
  const char * clause_name;
  unsigned num_clauses = pragma_stmt_->getNumClauses();
  
  for(unsigned i = 0; i < num_clauses; i ++) {
    omp_clause = pragma_stmt_->getClause(i);
    clause_name = getOpenMPClauseName(omp_clause->getClauseKind());
    VarList_ *var_list = new VarList_;

    if(strcmp(clause_name, "shared") == 0 || strcmp(clause_name, "private") == 0 || strcmp(clause_name, "firstprivate") == 0) {
      
      for(clang::StmtRange stmt_range = omp_clause->children(); stmt_range; ++ stmt_range) {
        const clang::DeclRefExpr *decl_ref_expr = static_cast<const clang::DeclRefExpr *>(*stmt_range);
        if(decl_ref_expr) {
          const clang::NamedDecl *named_decl = decl_ref_expr->getFoundDecl();
          const clang::ValueDecl *value_decl = decl_ref_expr->getDecl();
          var_list->insert(std::pair<std::string, std::string>(named_decl->getNameAsString(), value_decl->getType().getAsString()));
        }
      }
    }else if(strcmp(clause_name, "period") == 0) {

      clang::OMPPeriodClause *omp_peroid_clause = static_cast<clang::OMPPeriodClause *>(omp_clause);
      const clang::IntegerLiteral *int_literal = static_cast<const clang::IntegerLiteral *>(omp_peroid_clause->getPeriodValue());
      char period_val[100];
      sprintf(period_val, "%lu", int_literal->getValue().getZExtValue());
      var_list->insert(std::pair<std::string, std::string>(period_val, ""));
    }else {
      var_list->insert(std::pair<std::string, std::string>("", ""));
    }

    option_vect_->insert(std::pair<std::string, VarList_>(clause_name, *var_list));
  }
}

void Node::CreateXMLPragmaNode(tinyxml2::XMLDocument *xml_doc, tinyxml2::XMLElement *pragmas_element) {
  
  tinyxml2::XMLElement *pragma_element = xml_doc->NewElement("Pragma");
  pragmas_element->InsertEndChild(pragma_element);

  tinyxml2::XMLElement *name_element = xml_doc->NewElement("Name");
  pragma_element->InsertEndChild(name_element);
  
  tinyxml2::XMLText* name_text = xml_doc->NewText(pragma_type_.c_str());
  name_element->InsertEndChild(name_text);

  tinyxml2::XMLElement *position_element = xml_doc->NewElement("Position");
  
  if(option_vect_->size() != 0) {
    tinyxml2::XMLElement *options_element = xml_doc->NewElement("Options");
    pragma_element->InsertEndChild(options_element);

    CreateXMLPragmaOptions(xml_doc, options_element);

    pragma_element->InsertAfterChild(options_element, position_element);
  } else {

/*
 * ---- Position ----
 */  
    pragma_element->InsertEndChild(position_element);
  }

  tinyxml2::XMLElement *start_line_element = xml_doc->NewElement("StartLine");
  position_element->InsertEndChild(start_line_element);
  char start_line[100];
  sprintf(start_line, "%d", start_line_);
  tinyxml2::XMLText* start_line_text = xml_doc->NewText(start_line);
  start_line_element->InsertEndChild(start_line_text);

  tinyxml2::XMLElement *end_line_element = xml_doc->NewElement("EndLine");
  position_element->InsertEndChild(end_line_element);
  char end_line[100];
  sprintf(end_line, "%d", end_line_);
  tinyxml2::XMLText* end_line_text = xml_doc->NewText(end_line);
  end_line_element->InsertEndChild(end_line_text);

/*
 * ----- If present insert info of the For stmt ---- 
 */
  if(for_node_) {
    tinyxml2::XMLElement *for_element = xml_doc->NewElement("For");
    pragma_element->InsertEndChild(for_element);
    for_node_->CreateXMLPragmaFor(xml_doc, for_element);
  }

  if(children_vect_->size() != 0) {
    tinyxml2::XMLElement *nesting_element = xml_doc->NewElement("Children");
    pragma_element->InsertEndChild(nesting_element);
    tinyxml2::XMLElement *new_pragmas_element = xml_doc->NewElement("Pragmas");
    nesting_element->InsertEndChild(new_pragmas_element);
    for(std::vector<Node *>::iterator node_itr = children_vect_->begin(); node_itr != children_vect_->end(); ++node_itr) {
      (*node_itr)->CreateXMLPragmaNode(xml_doc, new_pragmas_element);
    }
  }
}

void Node::CreateXMLPragmaOptions(tinyxml2::XMLDocument *xml_doc, tinyxml2::XMLElement *options_element) {
  if(option_vect_->size() != 0) {
    
    for(std::map<std::string, VarList_>::iterator options_itr = option_vect_->begin(); options_itr != option_vect_->end(); ++ options_itr) {

      tinyxml2::XMLElement *option_element = xml_doc->NewElement("Option");
      options_element->InsertEndChild(option_element);

      tinyxml2::XMLElement *option_name_element = xml_doc->NewElement("Name");
      option_element->InsertEndChild(option_name_element);
      tinyxml2::XMLText* name_opt_text = xml_doc->NewText((*options_itr).first.c_str());
      option_name_element->InsertEndChild(name_opt_text);

      if((*options_itr).second.size() != 0) {
        for(std::map<std::string, std::string>::iterator var_itr = (*options_itr).second.begin(); var_itr != (*options_itr).second.end(); ++ var_itr) {
          tinyxml2::XMLElement *parameter_element = xml_doc->NewElement("Parameter");
          option_element->InsertEndChild(parameter_element);

          if(strcmp((*var_itr).first.c_str(), "") != 0) {
            tinyxml2::XMLElement *type_element = xml_doc->NewElement("Type");
            tinyxml2::XMLText* type_text = xml_doc->NewText((*var_itr).second.c_str());
            type_element->InsertEndChild(type_text);
            parameter_element->InsertEndChild(type_element);
          }
          tinyxml2::XMLElement *name_element = xml_doc->NewElement("Var");
          tinyxml2::XMLText* name_text = xml_doc->NewText((*var_itr).first.c_str());
          name_element->InsertEndChild(name_text);
          parameter_element->InsertEndChild(name_element);

        }
      }
    }
  }
}
\end{lstlisting}

\begin{lstlisting}[language=CCC, caption=pragma\_handler/ForNode.cpp]
#include "pragma_handler/ForNode.h"


ForNode::ForNode(clang::ForStmt *for_stmt) {
  loop_var_type_ = "";
  loop_var_init_val_set_ = false;

  loop_var_init_var_ = "";

  condition_val_set_ = false;
  condition_var_ = "";

  increment_val_set_ = false;
  increment_var_ = "";

  for_stmt_ = for_stmt;
  ExtractForParameters(for_stmt);
}

void ForNode::ExtractForParameters(clang::ForStmt *for_stmt) {
  
  ExtractForInitialization(for_stmt);
  ExtractForCondition(for_stmt);
  ExtractForIncrement(for_stmt);

}

void ForNode::ExtractForInitialization(clang::ForStmt *for_stmt) {
/*
* Initialization of the loop variable
*/

  /* for(int i = ....) */
  if(strcmp(for_stmt->child_begin()->getStmtClassName(), "DeclStmt") == 0) {
    const clang::DeclStmt *decl_stmt = static_cast<const clang::DeclStmt *>(*(for_stmt->child_begin()));
    const clang::Decl *decl = decl_stmt->getSingleDecl();
    
    /* Return the name of the variable */
    const clang::NamedDecl *named_decl = static_cast<const clang::NamedDecl *>(decl);            
    loop_var_ = named_decl->getNameAsString();
    
    /* Return the type of the variable */  
    const clang::ValueDecl *vale_decl = static_cast<const clang::ValueDecl *>(named_decl);
    loop_var_type_ = vale_decl->getType().getAsString();

    /* for (... = 0) */
    if(strcmp(decl_stmt->child_begin()->getStmtClassName(), "IntegerLiteral") == 0) {      
      const clang::IntegerLiteral *int_literal = static_cast<const clang::IntegerLiteral *>(*(decl_stmt->child_begin())); 
      loop_var_init_val_ = int_literal->getValue().getZExtValue();
      loop_var_init_val_set_ = true;
    
    /* for (... = a) */
    }else if (strcmp(decl_stmt->child_begin()->getStmtClassName(), "ImplicitCastExpr") == 0) {
      const clang::DeclRefExpr *decl_ref_expr = static_cast<const clang::DeclRefExpr *>(*(decl_stmt->child_begin()->child_begin()));
      const clang::NamedDecl *named_decl = decl_ref_expr->getFoundDecl();
      loop_var_init_var_ = named_decl->getNameAsString();
    }

  /* for ( i = ...) */
  }else if(strcmp(for_stmt->child_begin()->getStmtClassName(), "BinaryOperator") == 0) {
    const clang::BinaryOperator *binary_op = static_cast<const clang::BinaryOperator *>(*(for_stmt->child_begin())); 
    const clang::DeclRefExpr *decl_ref_expr = static_cast<const clang::DeclRefExpr *>(*(binary_op->child_begin()));
    
    //Return the name of the variable
    const clang::NamedDecl *named_decl = decl_ref_expr->getFoundDecl();
    loop_var_ = named_decl->getNameAsString();
    
    /* for( ... = 0) */
    clang::ConstStmtIterator stmt_itr = binary_op->child_begin();
    stmt_itr ++;
    if(strcmp(stmt_itr->getStmtClassName(), "IntegerLiteral") == 0) {
      const clang::IntegerLiteral *int_literal = static_cast<const clang::IntegerLiteral *>(*stmt_itr);
      loop_var_init_val_ = int_literal->getValue().getZExtValue();      
      loop_var_init_val_set_ = true;      

    /* for ( ... = a) */
    } else if (strcmp(stmt_itr->getStmtClassName(), "ImplicitCastExpr") == 0) {
      const clang::DeclRefExpr *decl_ref_expr = static_cast<const clang::DeclRefExpr *>(*(stmt_itr->child_begin()));
      const clang::NamedDecl *named_decl = decl_ref_expr->getFoundDecl();
      loop_var_init_var_ = named_decl->getNameAsString();
    }
  }
}

void ForNode::ExtractForCondition(clang::ForStmt *for_stmt) {

  const clang::Expr *condition_expr = for_stmt->getCond();
  const clang::BinaryOperator *binary_op = static_cast<const clang::BinaryOperator *>(condition_expr);
  
  /* Conditional funcion */
  condition_op_ = binary_op->getOpcodeStr();

  /* Conditional value */
  const clang::Expr *right_expr = binary_op->getRHS();

  if(strcmp(right_expr->getStmtClassName(), "IntegerLiteral") == 0) {
    const clang::IntegerLiteral *int_literal = static_cast<const clang::IntegerLiteral *>(right_expr);
    condition_val_ = int_literal->getValue().getZExtValue();
    condition_val_set_ = true;

  }else if(strcmp(right_expr->getStmtClassName(), "ImplicitCastExpr") == 0) {
    const clang::DeclRefExpr *decl_ref_expr = static_cast<const clang::DeclRefExpr *>(*(right_expr->child_begin()));
    const clang::NamedDecl *named_decl = decl_ref_expr->getFoundDecl();

/*
 * ---- PROBLEM: If the variable is not defined inside the block (which block?)
 * ----          the NameDecl * is != NULL, but when you try to exctract the name -> segmentation fault!!
 */
    condition_var_ = named_decl->getNameAsString();
  
  }
}

void ForNode::ExtractForIncrement(clang::ForStmt *for_stmt) {
  
  const clang::Expr *increment_expr = for_stmt->getInc();

  if(strcmp(increment_expr->getStmtClassName(), "UnaryOperator") == 0) {
    const clang::UnaryOperator *unary_op =  static_cast<const clang::UnaryOperator *>(increment_expr);
    increment_op_ = unary_op->getOpcodeStr(unary_op->getOpcode());

  }else if(strcmp(increment_expr->getStmtClassName(), "CompoundAssignOperator") == 0) {
    const clang::CompoundAssignOperator *compound_op = static_cast<const clang::CompoundAssignOperator *>(increment_expr);
    increment_op_ = compound_op->getOpcodeStr();
    const clang::Expr *right_expr = compound_op->getRHS();

    if(strcmp(right_expr->getStmtClassName(), "IntegerLiteral") == 0) {
      const clang::IntegerLiteral *int_literal = static_cast<const clang::IntegerLiteral *>(right_expr);
      increment_val_ = int_literal->getValue().getZExtValue();
      increment_val_set_ = true;

    }else if(strcmp(right_expr->getStmtClassName(), "ImplicitCastExpr") == 0) {
      const clang::DeclRefExpr *decl_ref_expr = static_cast<const clang::DeclRefExpr *>(*(right_expr->child_begin()));
      const clang::NamedDecl *named_decl = decl_ref_expr->getFoundDecl();
      increment_var_ = named_decl->getNameAsString();  
    }
  }

}

void ForNode::CreateXMLPragmaFor(tinyxml2::XMLDocument *xml_doc, tinyxml2::XMLElement *for_element) {

/*
 * ----- DECLARATION -----
 */ 
  tinyxml2::XMLElement *declaration_element = xml_doc->NewElement("Declaration");
  for_element->InsertEndChild(declaration_element);

  tinyxml2::XMLElement *type_element = xml_doc->NewElement("Type");
  declaration_element->InsertEndChild(type_element);
  tinyxml2::XMLText* type_text = xml_doc->NewText(loop_var_type_.c_str());
  type_element->InsertEndChild(type_text);

  tinyxml2::XMLElement *loop_var_element = xml_doc->NewElement("LoopVariable");
  declaration_element->InsertEndChild(loop_var_element);
  tinyxml2::XMLText* loop_var_text = xml_doc->NewText(loop_var_.c_str());
  loop_var_element->InsertEndChild(loop_var_text);

  if(loop_var_init_val_set_ == true) {
    tinyxml2::XMLElement *init_val_element = xml_doc->NewElement("InitValue");
    declaration_element->InsertEndChild(init_val_element);
    char loop_var_init_val[100];
    sprintf(loop_var_init_val, "%d", loop_var_init_val_);
    tinyxml2::XMLText* init_val_text = xml_doc->NewText(loop_var_init_val);
    init_val_element->InsertEndChild(init_val_text);
  }else {
    tinyxml2::XMLElement *init_var_element = xml_doc->NewElement("InitVariable");
    declaration_element->InsertEndChild(init_var_element);
    tinyxml2::XMLText* init_var_text = xml_doc->NewText(loop_var_init_var_.c_str());
    init_var_element->InsertEndChild(init_var_text);
  }

/*
 * ---- CONDITION -----
 */ 
  tinyxml2::XMLElement *condition_element = xml_doc->NewElement("Condition");
  for_element->InsertAfterChild(declaration_element, condition_element); 
  
  tinyxml2::XMLElement *condition_op_element = xml_doc->NewElement("Op");
  condition_element->InsertEndChild(condition_op_element);
  tinyxml2::XMLText* condition_op_text = xml_doc->NewText(condition_op_.c_str());
  condition_op_element->InsertEndChild(condition_op_text);

  if(condition_val_set_ == true) {
    tinyxml2::XMLElement *condition_val_element = xml_doc->NewElement("ConditionValue");
    condition_element->InsertEndChild(condition_val_element);
    char condition_val[100];
    sprintf(condition_val, "%d", condition_val_);
    tinyxml2::XMLText* condition_val_text = xml_doc->NewText(condition_val);
    condition_val_element->InsertEndChild(condition_val_text);
  
  }else {
    tinyxml2::XMLElement *condition_var_element = xml_doc->NewElement("ConditionVariable");
    condition_element->InsertEndChild(condition_var_element);
    tinyxml2::XMLText* condition_var_text = xml_doc->NewText(condition_var_.c_str());
    condition_var_element->InsertEndChild(condition_var_text);
  }

/*
 * ---- INCREMENT ----
 */
  tinyxml2::XMLElement *increment_element = xml_doc->NewElement("Increment");
  for_element->InsertAfterChild(condition_element, increment_element); 

  tinyxml2::XMLElement *increment_op_element = xml_doc->NewElement("Op");
  increment_element->InsertEndChild(increment_op_element);
  tinyxml2::XMLText* increment_op_text = xml_doc->NewText(increment_op_.c_str());
  increment_op_element->InsertEndChild(increment_op_text);

  if(increment_val_set_ == true) {
    tinyxml2::XMLElement *increment_val_element = xml_doc->NewElement("IncrementValue");
    increment_element->InsertEndChild(increment_val_element);
    char increment_val[100];
    sprintf(increment_val, "%d", increment_val_);
    tinyxml2::XMLText* increment_val_text = xml_doc->NewText(increment_val);
    increment_val_element->InsertEndChild(increment_val_text);

  }else if(increment_var_.compare("") != 0) {
    tinyxml2::XMLElement *increment_var_element = xml_doc->NewElement("IncrementVariable");
    increment_element->InsertEndChild(increment_var_element);
    tinyxml2::XMLText* increment_var_text = xml_doc->NewText(increment_var_.c_str());
    increment_var_element->InsertEndChild(increment_var_text);
  }


}
\end{lstlisting}

\begin{lstlisting}[language=CCC, caption=pragma\_handler/Root.cpp]
#include "pragma_handler/Root.h"

Root::Root(Node *n, FunctionInfo funct_info) {

	children_vect_ = new std::vector<Node *>();
	children_vect_->push_back(n);

	last_node_ = n;
  function_info_ = funct_info;
}

void Root::CreateXMLFunction(tinyxml2::XMLDocument *xml_doc) {

  tinyxml2::XMLElement *function_element = xml_doc->NewElement("Function");
  xml_doc->LastChild()->InsertEndChild(function_element);

  tinyxml2::XMLElement *name_element = xml_doc->NewElement("Name");
  function_element->InsertEndChild(name_element); 
  tinyxml2::XMLText* name_text = xml_doc->NewText(function_info_.function_name_.c_str());
  name_element->InsertEndChild(name_text);

  if(function_info_.function_class_name_.compare("") != 0){
    tinyxml2::XMLElement *class_name_element = xml_doc->NewElement("ClassName");
    function_element->InsertEndChild(class_name_element); 
    tinyxml2::XMLText* class_name_text = xml_doc->NewText(function_info_.function_class_name_.c_str());
    class_name_element->InsertEndChild(class_name_text);
  }
    
  tinyxml2::XMLElement *return_type_element = xml_doc->NewElement("ReturnType");
  function_element->InsertEndChild(return_type_element);
  tinyxml2::XMLText* return_type_text = xml_doc->NewText(function_info_.function_return_type_.c_str());
  return_type_element->InsertEndChild(return_type_text);

  if(function_info_.num_params_ > 0) {
    tinyxml2::XMLElement *parameters_element = xml_doc->NewElement("Parameters");
    function_element->InsertEndChild(parameters_element);

    for(int i = 0; i < function_info_.num_params_; i ++) {
      tinyxml2::XMLElement *parameter_element = xml_doc->NewElement("Parameter");
      parameters_element->InsertEndChild(parameter_element);

      tinyxml2::XMLElement *type_element = xml_doc->NewElement("Type");
      parameter_element->InsertEndChild(type_element);
      tinyxml2::XMLText* param_type_text = xml_doc->NewText(function_info_.function_parameters_[i][0].c_str());
      type_element->InsertEndChild(param_type_text);

      tinyxml2::XMLElement *param_name_element = xml_doc->NewElement("Name");
      parameter_element->InsertEndChild(param_name_element);
      tinyxml2::XMLText* param_name_text = xml_doc->NewText(function_info_.function_parameters_[i][1].c_str());
      param_name_element->InsertEndChild(param_name_text);

    }
  }

  tinyxml2::XMLElement *line_element = xml_doc->NewElement("Line");
  function_element->InsertEndChild(line_element); 
  char line[100];
  sprintf(line, "%d", function_info_.function_start_line_);
  tinyxml2::XMLText* line_text = xml_doc->NewText(line);
  line_element->InsertEndChild(line_text);


  tinyxml2::XMLElement *pragmas_element = xml_doc->NewElement("Pragmas");
  function_element->InsertEndChild(pragmas_element);

  for(std::vector<Node *>::iterator node_itr = children_vect_->begin(); node_itr != children_vect_->end(); ++ node_itr) {
    (*node_itr)->CreateXMLPragmaNode(xml_doc, pragmas_element);
  }
}
\end{lstlisting}

\begin{lstlisting}[language=CCC, caption=pragma\_handler/create\_tree.cpp]
#include "pragma_handler/create_tree.h"

std::vector<Root *> *CreateTree(std::vector<clang::OMPExecutableDirective *> *pragma_list,
								std::vector<clang::FunctionDecl *> *function_list, clang::SourceManager &sm) {

	clang::FunctionDecl *function_decl = NULL;
  clang::FunctionDecl *function_decl_tmp = NULL;
	std::vector<Root *> *root_vect = new std::vector<Root *>();

  std::vector<clang::OMPExecutableDirective *>::iterator omp_itr;

	for(omp_itr = pragma_list->begin(); omp_itr != pragma_list->end(); ++ omp_itr) {    

    function_decl_tmp = GetFunctionForPragma(*omp_itr, function_list, sm);
    Node * n = new Node(*omp_itr, function_decl_tmp, sm);

    /* In case of parallel for skip one stmt. 
       Parallel for is represented with two OMPExecutableDirective,
       (OMPParallel + OMPFor) so we have to skip one stmt */
    if((*omp_itr)->getAssociatedStmt()) {
      if(strcmp((*omp_itr)->getStmtClassName(), "OMPParallelDirective") == 0 
          && utils::Line((*omp_itr)->getAssociatedStmt()->getLocStart(), sm) 
             == utils::Line((*omp_itr)->getAssociatedStmt()->getLocEnd(), sm)) {
        n->pragma_type_ = "OMPParallelForDirective";
        omp_itr++;
      }
    }

    if(function_decl_tmp != function_decl) {
      function_decl = function_decl_tmp;
      Root *root = new Root(n, n->getParentFunctionInfo());
      n->setParentNode(NULL);
      root->setLastNode(n);
      root_vect->push_back(root);

    }else {
      BuildTree(root_vect->back(), n);
      root_vect->back()->setLastNode(n);
    }
  }
  return root_vect;
}


clang::FunctionDecl *GetFunctionForPragma(clang::OMPExecutableDirective *pragma_stmt, 
										  std::vector<clang::FunctionDecl *> *function_list, 
										  clang::SourceManager &sm) {

  unsigned funct_start_line, funct_end_line;
	unsigned pragma_start_line = utils::Line(pragma_stmt->getLocStart(), sm);
	std::vector<clang::FunctionDecl *>::iterator funct_itr;

	for(funct_itr = function_list->begin(); funct_itr != function_list->end(); ++ funct_itr) {
		funct_start_line = utils::Line((*funct_itr)->getSourceRange().getBegin(), sm);
		funct_end_line = utils::Line((*funct_itr)->getSourceRange().getEnd(), sm);
		if(pragma_start_line < funct_end_line && pragma_start_line > funct_start_line)
			return (*funct_itr);
	}
	return NULL;
}


/*
 * ---- Attach the node to the correct parent (if the node is node annidated attach it to root) ----
 * THEOREM: A node can be annidated only in its previous node or in the father of the previous node or in the father 
 *          of the father ..... of the previous node. (This is due to the fact that the list of pragmas is ordered based
            on starting line of the associated stmt).
 */ 
void BuildTree(Root *root, Node *n) {

  Node *last_node = root->getLastNode();
  bool annidation;

  while(last_node != NULL) {
    annidation = CheckAnnidation(last_node, n);

    if(annidation == true) {
      last_node->AddChildNode(n);
      n->setParentNode(last_node);
      return;

    }else 
      last_node = last_node->getParentNode();
  }

  root->AddChildNode(n);
  n->setParentNode(NULL);
}

/*
 * ---- Check if n is annidated inside parent: to be annidated it is enough that n->endLine < parent->endLine 
 * (for sure n->startLine < parent->startLine because pragmas are ordered based on their starting line)
 */ 
bool CheckAnnidation(Node *parent, Node *n) {

  if(n->getEndLine() < parent->getEndLine())  
    return true;
  else
    return false;
  
}
\end{lstlisting}

\begin{lstlisting}[language=CCC, caption=utilis/source\_locations.cpp]
#include "utils/source_locations.h"

using namespace std;
using namespace clang;

namespace utils {

string FileName(SourceLocation const& l, SourceManager const& sm) {
	PresumedLoc pl = sm.getPresumedLoc(l);
	return string(pl.getFilename());
}

string FileId(SourceLocation const& l, SourceManager const& sm) {
	string fn = FileName(l, sm);
	for(size_t i=0; i<fn.length(); ++i)
		switch(fn[i]) {
			case '/':
			case '\\':
			case '>':
			case '.':
				fn[i] = '_';
		}
	return fn;
}

unsigned Line(SourceLocation const& l, SourceManager const& sm) {
	PresumedLoc pl = sm.getPresumedLoc(l);
	return pl.getLine();
}

std::pair<unsigned, unsigned> Line(clang::SourceRange const& r, SourceManager const& sm) {
	return std::make_pair(Line(r.getBegin(), sm), Line(r.getEnd(), sm));
}

unsigned Column(SourceLocation const& l, SourceManager const& sm) {
	PresumedLoc pl = sm.getPresumedLoc(l);
	return pl.getColumn();
}

std::pair<unsigned, unsigned> Column(clang::SourceRange const& r, SourceManager const& sm) {
	return std::make_pair(Column(r.getBegin(), sm), Column(r.getEnd(), sm));
}

std::string location(clang::SourceLocation const& l, clang::SourceManager const& sm) {
	std::string str;
	llvm::raw_string_ostream ss(str);
	l.print(ss,sm);
	return ss.str();
}
}
\end{lstlisting}

\begin{lstlisting}[language=CCC, caption=xml\_creator/xml\_creator.cpp]
#include "xml_creator/XMLcreator.h"

void CreateXML(std::vector<Root *> *root_vect, char *file_name) {
  
  tinyxml2::XMLDocument *xml_doc = new tinyxml2::XMLDocument();
  tinyxml2::XMLElement *file_element = xml_doc->NewElement("File");
  xml_doc->InsertEndChild(file_element);

  tinyxml2::XMLElement *name_element = xml_doc->NewElement("Name");
  tinyxml2::XMLText* name_text = xml_doc->NewText(file_name);
  name_element->InsertEndChild(name_text);
  file_element->InsertEndChild(name_element);


  for(std::vector<Root *>::iterator root_itr = root_vect->begin(); root_itr != root_vect->end(); ++ root_itr)     
    (*root_itr)->CreateXMLFunction(xml_doc);
  

  std::string out_xml_file (file_name);
  size_t ext = out_xml_file.find_last_of(".");
  if (ext == std::string::npos)
    ext = out_xml_file.length();
  out_xml_file = out_xml_file.substr(0, ext);
  std::cout << out_xml_file << std::endl;
  
  out_xml_file.insert(ext, "_pragmas.xml");
  std::cout << out_xml_file << std::endl;

  xml_doc->SaveFile(out_xml_file.c_str());
}
\end{lstlisting}


\section{Run-time}
\subsection{Profiler}
\begin{lstlisting}[language=CCC, caption=profile\_tracker.h]
#include <fstream>
#include <time.h>     
#include <iostream>
#include <unistd.h>

#define log_file "log_file.xml"

struct ProfileTrackParams {

	ProfileTrackParams(int funct_id, int pragma_line)
		: funct_id_(funct_id) ,pragma_line_(pragma_line), num_for_iteration_set_(false) {}
	/* Costructor for parallel for */
	ProfileTrackParams(int funct_id, int pragma_line, int n) 
		: funct_id_(funct_id), pragma_line_(pragma_line), num_for_iteration_(n), num_for_iteration_set_(true) {}

	int funct_id_; 
	int pragma_line_;
	/* In the case of a parallel for this variable saves the number of the iteration of the for */
	int num_for_iteration_;
	bool num_for_iteration_set_;
};

/*
 * ----- Class that keep track of the children time and the father of the current pragma in execution ----
 */
class ProfileTracker {

	clock_t start_time_;
	clock_t end_time_;

	int num_for_iteration_;
	bool num_for_iteration_set_;

	/* These functions print the result of the profiling in a log file */
	void PrintPragma();
	void PrintFunction();

public:
	int pragma_line_;
	int funct_id_;

	double elapsed_time_;
	/* Time spent by the children of the current pragma or function */
	double children_elapsed_time_;

	/* Keeps track of which function/pragma has invoked the current function/pragma */
	ProfileTracker *previous_pragma_executed_;

	/* In the costructor a timer is started */
	ProfileTracker(const ProfileTrackParams & p);
	/* In the destructor the timer is stopped and the elapsed time is written in the log file */
	~ProfileTracker();

	/* This is necessary to allow to create an object inside the declaration of an if stmt */
	operator bool() const  { return true; }
};

/*
 * ---- Singleton class that open and close the log file -----
 */
class ProfileTrackerLog {

	/* Keeps track of which function/pragma has invoked the current function/pragma */
	ProfileTracker *current_pragma_executing_;

	/* Create the log file and write in it the hardware spec */
	ProfileTrackerLog ();

	void WriteArchitecturesSpec();
	size_t getTotalSystemMemory();    

public:
	/* File where the log is written */
	std::ofstream log_file_;

	static ProfileTrackerLog* getInstance();
	/* Substitute the pointer of the current pragma in execution and return the previous value */
	ProfileTracker *ReplaceCurrentPragma(ProfileTracker *current_pragma_executing_);

	/* Save and close the log file */
	~ProfileTrackerLog();

};
\end{lstlisting}

\begin{lstlisting}[language=CCC, caption=profile\_tracker.cpp]
#include "profile_tracker/profile_tracker.h"


/*
 * ---- PROFILE TRACKER LOG ----
 */
ProfileTrackerLog::ProfileTrackerLog () {
	current_pragma_executing_ = NULL;
	log_file_.open(log_file);
	log_file_ << "<LogFile>" << std::endl;
	WriteArchitecturesSpec();
}

void ProfileTrackerLog::WriteArchitecturesSpec() {
	log_file_ << "	<Hardware ";
	log_file_ << "NumberofCores=\"" << std::thread::hardware_concurrency() << "\" ";
  	log_file_ << "MemorySize=\"" << getTotalSystemMemory() << "\"/>" << std::endl;
}	

size_t ProfileTrackerLog::getTotalSystemMemory() {
   	/*long pages = sysconf(_SC_PHYS_PAGES);
   	long page_size = sysconf(_SC_PAGE_SIZE);
   	return (pages * page_size)/1024/1024;*/
   	return 2000;
}

ProfileTrackerLog* ProfileTrackerLog::getInstance() {
    static ProfileTrackerLog log;
    return &log;
}

ProfileTrackerLog::~ProfileTrackerLog() {
	log_file_ << "</LogFile>" << std::endl;
	log_file_.close();
}

ProfileTracker *ProfileTrackerLog::ReplaceCurrentPragma(ProfileTracker *current_pragma_executing) {
	ProfileTracker *tmp = current_pragma_executing_;
	current_pragma_executing_ = current_pragma_executing;
	return tmp;
}


/* 
 * ---- PROFILE TRACKER ----
 */
ProfileTracker::ProfileTracker(const ProfileTrackParams & p) {
	previous_pragma_executed_ = ProfileTrackerLog::getInstance()->ReplaceCurrentPragma(this);

	children_elapsed_time_ = 0;	

	pragma_line_ = p.pragma_line_;
	funct_id_ = p.funct_id_;
	num_for_iteration_set_ = p.num_for_iteration_;

	if(num_for_iteration_set_)
		num_for_iteration_ = p.num_for_iteration_;

	start_time_ = clock();
}

ProfileTracker::~ProfileTracker() {
	end_time_ = clock();
	elapsed_time_ = ((double)(end_time_ - start_time_))/CLOCKS_PER_SEC;
	if(previous_pragma_executed_) {
		previous_pragma_executed_->children_elapsed_time_ += elapsed_time_;
	}	
	ProfileTrackerLog::getInstance()->ReplaceCurrentPragma(previous_pragma_executed_);

	if(pragma_line_ == 0)
		PrintFunction();
	else
		PrintPragma();

}
 
void ProfileTracker::PrintPragma() {
	ProfileTrackerLog::getInstance()->log_file_ << "	<Pragma" \
											 << " fid=\"" << funct_id_ << "\" pid=\"" << pragma_line_ << "\" ";
	if(previous_pragma_executed_) {
		if(previous_pragma_executed_->pragma_line_ != 0)
			ProfileTrackerLog::getInstance()->log_file_ << "callerid=\"" << previous_pragma_executed_->pragma_line_ << "\" "; 
		else 
			ProfileTrackerLog::getInstance()->log_file_ << "callerid=\"" << previous_pragma_executed_->funct_id_ << "\" "; 
	}
	ProfileTrackerLog::getInstance()->log_file_ << "elapsedTime=\"" << elapsed_time_ << "\" " \
											   << "childrenTime=\"" << children_elapsed_time_ << "\"";
	if(num_for_iteration_set_)
	 	ProfileTrackerLog::getInstance()->log_file_ << " loops=\"" << num_for_iteration_ << "\"";
	ProfileTrackerLog::getInstance()->log_file_ << "/>" << std::endl; 

}

void ProfileTracker::PrintFunction() {
	ProfileTrackerLog::getInstance()->log_file_ << "	<Function" \
												 << " fid=\"" << funct_id_ << "\" ";
	if(previous_pragma_executed_) {
		if(previous_pragma_executed_->pragma_line_ != 0)
			ProfileTrackerLog::getInstance()->log_file_ << "callerid=\"" << previous_pragma_executed_->pragma_line_ << "\" "; 
		else 
			ProfileTrackerLog::getInstance()->log_file_ << "callerid=\"" << previous_pragma_executed_->funct_id_ << "\" "; 
	}
	ProfileTrackerLog::getInstance()->log_file_ << "elapsedTime=\"" << elapsed_time_ <<  "\" " \
											   << "childrenTime=\"" << children_elapsed_time_ << "\"/>" << std::endl; 


}
\end{lstlisting}
\subsection{Final exectution}
\begin{lstlisting}[language=CCC, caption=thread\_pool.h]
#include <string>
#include <thread>
#include <vector>
#include <mutex>
#include <map>
#include <math.h>
#include <iostream>
#include <condition_variable>
#include <queue>
#include <exception>
#include <sys/time.h>

#include "xml_creator/tinyxml2.h"

int chartoint(const char *cc);
int chartoint(char *cc);

class ForParameter {
public:
    const int thread_id_;
    const int num_threads_;
    ForParameter(int thread_id, int num_threads) : thread_id_(thread_id), num_threads_(num_threads) {}
};

class NestedBase {
public:

    NestedBase(int pragma_id) : pragma_id_(pragma_id) {}
    
    int pragma_id_;
    std::queue<std::shared_ptr<NestedBase>> todo_job_;

    void launch_todo_job() {
        while(todo_job_.size() != 0) {
            todo_job_.front()->callme(ForParameter(0, 1));
            todo_job_.pop();
        }
    }
    
    virtual void callme(ForParameter for_param) = 0;
    virtual std::shared_ptr<NestedBase> clone() const = 0;
};

class ThreadPool {
public:
    typedef int Jobid_t;

    struct Job
    {
        std::shared_ptr<NestedBase> nested_base_;
        ForParameter for_param_;
        Job(std::shared_ptr<NestedBase> nested_base, ForParameter for_param)
            : nested_base_(nested_base), for_param_(for_param) {}
    };

    /* Launches the threads */
    void init(int pool_size);

    /* Called by the task to be put in the job queue */
    bool call(std::shared_ptr<NestedBase> nested_base);
    void call_sections(std::shared_ptr<NestedBase> nested_b);
    void call_parallel(std::shared_ptr<NestedBase> nested_b);
    void call_for(std::shared_ptr<NestedBase> nested_b);
    void call_barrier(std::shared_ptr<NestedBase> nested_b);

    /* Push a job in the job queue */
    void push(std::shared_ptr<NestedBase> nested_base, ForParameter for_param, int thread_id);
    void push_completed_job(std::shared_ptr<NestedBase> nested_base, ForParameter for_param);
    void push_termination_job(int thread_id);

    /* Pause a thread till the job[job_id] complete */
    void join(Jobid_t job_id);

    void joinall();

    static ThreadPool* getInstance(std::string file_name);

    /* Map the thread::id to an integer going from 0 to num_thread - 1 */
    std::map<std::thread::thread::id, int> thread_id_to_int_;

    ~ThreadPool() { joinall(); }

private:
    struct ScheduleOptions {
        int pragma_id_;
        int caller_id_;
        /* In case of a parallel for, specify to the job which part of the for to execute */
        int thread_id_;
        /* Idicates the pragma type: parallel, task, ... */
        std::string pragma_type_;
        /* Indicates the threads that have to run the task */
        std::vector<int> threads_;
        /* List of pragma_id_ to wait before completing the task */
        std::vector<int> barriers_;
    };

    struct JobIn {
        Job job_;
        /* ID of the job = pragma line number */
        Jobid_t job_id_;
        Jobid_t pragma_id_;
        /* Pragma type, e.g. OMPParallelDirective, OMPTaskDirective, ... */
        std::string job_type_;
        /* Fix the bug where a thread waits for another thread which already nofied to have compleated */
        bool job_completed_ = false;

        bool terminated_with_exceptions_ = false;

        std::unique_ptr<std::condition_variable> done_cond_var_;

        std::vector<int> barriers_;

        JobIn(std::shared_ptr<NestedBase> nested_base, ForParameter for_param)
                : job_(nested_base, for_param), job_completed_(false) {}

    };

    struct JobQueue {
        Jobid_t j_id_;
        int thread_id_;
        JobQueue(Jobid_t j_id, int thread_id) : j_id_(j_id), thread_id_(thread_id) {}
    };

    ThreadPool(std::string file_name);
    
    void run(int id);

    std::map<int, ScheduleOptions> sched_opt_;

    std::vector<std::thread> threads_pool_; // not thread safe
    
    /* Job queue for each thread */
    std::map<int, std::queue<JobQueue>> work_queue_;
    
    /* For each pragma the list of jobs executing that pragma, e.g. in case of parallel for */
    //typedef std::pair<Jobid_t, std::thread::id> JobKey;
    std::map<int, std::vector<JobIn>> known_jobs_;
    //std::map<int, std::map<int, JobIn>> known_jobs_;
    /* Mutex used by std::condition_variable to synchronize jobs execution */
    //std::mutex cond_var_mtx;
    std::map<std::thread::id, std::mutex> cond_var_mtx;
    std::mutex job_pop_mtx;
    std::mutex job_end;
};

\end{lstlisting}

\begin{lstlisting}[language=CCC, caption=thread\_pool.cpp]
/*
* In case of a parallel pragma is known that each pragma present in the parallel's barrier list has been
* invoked by the thread that runs the parallel pragma.
*
* In case of a barrier pragma is known that each pragma present in the barrier's barrier list has been invoked
* by the same thread that invoked the barrieri pragma.
*/


#include "threads_pool.h"


std::mutex singleton_mtx;


ThreadPool* ThreadPool::getInstance(std::string file_name) {
    singleton_mtx.lock();
    static ThreadPool thread_pool(file_name);
    singleton_mtx.unlock();
    return &thread_pool;
}


ThreadPool::ThreadPool(std::string file_name) {
    /* Create schdule xml file name from source code file name, e.g. test.cpp -> test_schedule.xml*/
    std::string in_xml_file (file_name);
    size_t ext = in_xml_file.find_last_of(".");
    if (ext == std::string::npos)
        ext = in_xml_file.length();
    in_xml_file = in_xml_file.substr(0, ext);
    in_xml_file.insert(ext, "_schedule.xml");

    tinyxml2::XMLDocument xml_doc;
    //xml_doc.LoadFile(in_xml_file.c_str());
    xml_doc.LoadFile("schedule.xml");

    tinyxml2::XMLElement *threads_num_element = xml_doc.FirstChildElement("Schedule")->FirstChildElement("Cores");

    const char* threads_num = threads_num_element->GetText();
    /* Set the number of thread as the number of cores plus one thread wich is used to run parallel and sections job */
    init(chartoint(threads_num));

    tinyxml2::XMLElement *pragma_element = xml_doc.FirstChildElement("Schedule")->FirstChildElement("Pragma");
    while(pragma_element != NULL) {
        ScheduleOptions sched_opt;

        const char* pragma_id = pragma_element->FirstChildElement("id")->GetText();
        int id = chartoint(pragma_id);
        sched_opt.pragma_id_ = id;
        

        tinyxml2::XMLElement *pragma_type_element = pragma_element->FirstChildElement("Type");
        const char* pragma_type = pragma_type_element->GetText();
        sched_opt.pragma_type_ = pragma_type;

        tinyxml2::XMLElement *thread_element = pragma_element->FirstChildElement("Threads");
        if(thread_element != NULL)
            thread_element = thread_element->FirstChildElement("Thread");
        
        while(thread_element != NULL){
            const char *thread_id = thread_element->GetText();
            sched_opt.threads_.push_back(chartoint(thread_id));

            thread_element = thread_element->NextSiblingElement("Thread");
        }
        
        tinyxml2::XMLElement *barriers_element = pragma_element->FirstChildElement("Barrier");
        if(barriers_element != NULL)
            barriers_element = barriers_element->FirstChildElement("id");
        while(barriers_element != NULL){
            const char *thread_id = barriers_element->GetText();
            sched_opt.barriers_.push_back(chartoint(thread_id));

            barriers_element = barriers_element->NextSiblingElement("id");
        }

        sched_opt_[id] = sched_opt;
        pragma_element = pragma_element->NextSiblingElement("Pragma");
    }
        //for(std::map<int, ScheduleOptions>::iterator itr = sched_opt_.begin(); itr != sched_opt_.end(); ++ itr)
        //std::cout << "Pragma id: " << (*itr).second.pragma_id_ << ", type: " << (*itr).second.pragma_type_ << std::endl;

}


void ThreadPool::init(int pool_size)
{
    /* This is needed cause otherwise the main process would be considered as thread num 0*/
    thread_id_to_int_[std::this_thread::get_id()] = -1;
    //std::cout << std::this_thread::get_id() << " = -1 " << std::endl;

    threads_pool_.reserve(pool_size);
    for(int i = 0; i < pool_size; i++) {
        threads_pool_.push_back(std::thread(&ThreadPool::run,this, i));
    }

}

/* If a job has to allocate a job on its own thread, it first allocates all other job and then execute directly that job */
/* This solve the problem of a parallel for. */
bool ThreadPool::call(std::shared_ptr<NestedBase> nested_b) {
    int thread_number = sched_opt_[nested_b->pragma_id_].threads_.size();
    int thread_id;
    /* Get the integer id of the running thread */
    int my_id = thread_id_to_int_[std::this_thread::get_id()];
    /* In case of a parallel for */

    if(sched_opt_[nested_b->pragma_id_].pragma_type_.compare("OMPForDirective") == 0
        || sched_opt_[nested_b->pragma_id_].pragma_type_.compare("OMPParallelForDirective") == 0) {
        
        call_for(nested_b);

    }else {
        thread_id = sched_opt_[nested_b->pragma_id_].threads_[0];
        if(thread_id != my_id) {
            push(nested_b->clone(), ForParameter(0, 1), thread_id);
            
            if(sched_opt_[nested_b->pragma_id_].pragma_type_.compare("OMPParallelDirective") == 0) {
                int barriers_id = sched_opt_[nested_b->pragma_id_].barriers_[0];

                join(barriers_id);
                
                int thread_num = sched_opt_[nested_b->pragma_id_].threads_[0];
                std::thread::id t_id = threads_pool_[thread_num].get_id();
                int barriers_number = sched_opt_[nested_b->pragma_id_].barriers_.size();
                
                for (int i = 1; i < barriers_number; i ++) {
                    barriers_id = sched_opt_[nested_b->pragma_id_].barriers_[i];
                    join(barriers_id);
                }
            }
        }else {
            if(sched_opt_[nested_b->pragma_id_].pragma_type_.compare("OMPParallelDirective") == 0)
                call_parallel(nested_b);
            else if (sched_opt_[nested_b->pragma_id_].pragma_type_.compare("OMPSectionsDirective") == 0
                || sched_opt_[nested_b->pragma_id_].pragma_type_.compare("OMPSingleDirective") == 0)
                call_sections(nested_b);
            else if(sched_opt_[nested_b->pragma_id_].pragma_type_.compare("OMPBarrierDirective") == 0)
                call_barrier(nested_b);
            else {
                push_completed_job(nested_b, ForParameter(0, 1));
                return true;
            }
        }

    }
    return false;
}

void ThreadPool::call_sections(std::shared_ptr<NestedBase> nested_b){
    nested_b->callme(ForParameter(0, 1));
    
    int barriers_number = sched_opt_[nested_b->pragma_id_].barriers_.size();
    int barrier_id;
    for(int i = 0; i < barriers_number; i ++) {
        barrier_id = sched_opt_[nested_b->pragma_id_].barriers_[i];
        join(barrier_id);
    }
    push_completed_job(nested_b, ForParameter(0, 1));
}

void ThreadPool::call_parallel(std::shared_ptr<NestedBase> nested_b) {
    nested_b->callme(ForParameter(0, 1));

    if(sched_opt_[nested_b->pragma_id_].pragma_type_.compare("OMPParallelDirective") == 0) {
        int barriers_id = sched_opt_[nested_b->pragma_id_].barriers_[0];
        join(barriers_id);
            
        int barriers_number = sched_opt_[nested_b->pragma_id_].barriers_.size();
        for (int i = 1; i < barriers_number; i ++) {
            barriers_id = sched_opt_[nested_b->pragma_id_].barriers_[i];
            int thread_num = sched_opt_[barriers_id].threads_[0];
            std::thread::id t_id = threads_pool_[thread_num].get_id();
            join(barriers_id);
        }
    }
}

void ThreadPool::call_for(std::shared_ptr<NestedBase> nested_b) {
    int thread_number = sched_opt_[nested_b->pragma_id_].threads_.size();
    int thread_id;
    /* Get the integer id of the running thread */
    int my_id = thread_id_to_int_[std::this_thread::get_id()];
    for(int i = 0; i < thread_number; i ++) {
        thread_id = sched_opt_[nested_b->pragma_id_].threads_[i];
        if(thread_id != my_id) {
            push(nested_b->clone(), ForParameter(i, thread_number), thread_id);
        }
    }
    /* If a son and a father are on the same thread!!! */
    for(int i = 0; i < thread_number; i ++) {
        thread_id = sched_opt_[nested_b->pragma_id_].threads_[i];
        if(thread_id == my_id) {
            push_completed_job(nested_b->clone(), ForParameter(i, thread_number));
            nested_b->callme(ForParameter(i, thread_number));
        }
    }

    //if(sched_opt_[nested_b->pragma_id_].barriers_.size() > 0) {
        int barriers_id = sched_opt_[nested_b->pragma_id_].barriers_[0];
        join(barriers_id);
    //}
}

void ThreadPool::call_barrier(std::shared_ptr<NestedBase> nested_b) {
    int barriers_number = sched_opt_[nested_b->pragma_id_].barriers_.size();
    int barriers_id, threads_num;
    for (int i = 0; i < barriers_number; i ++) {
        barriers_id = sched_opt_[nested_b->pragma_id_].barriers_[i];
        join(barriers_id);
    }
}

/* Insert a job wich has the flag completed already setted. This is necessary in case a thread executes more
job consecutively */
void ThreadPool::push_completed_job(std::shared_ptr<NestedBase> nested_base,
                                     ForParameter for_param) {

    Jobid_t id = nested_base->pragma_id_;
        
    JobIn job_in(nested_base, for_param);
    job_in.job_id_ = id;
    job_in.job_completed_ = true;

    job_pop_mtx.lock();
    if(known_jobs_[id].size() == 0)
        known_jobs_[id].reserve(for_param.num_threads_);
    known_jobs_[id].push_back(std::move(job_in));
    job_pop_mtx.unlock();
}


void ThreadPool::push(std::shared_ptr<NestedBase> nested_base,
                                     ForParameter for_param, int thread_id) {
    
    Jobid_t id = nested_base->pragma_id_;
        
    JobIn job_in(nested_base, for_param);
    job_in.job_id_ = id;
    job_in.job_type_ = sched_opt_[nested_base->pragma_id_].pragma_type_;
    job_in.done_cond_var_ =
            std::unique_ptr<std::condition_variable>(new std::condition_variable());
       
    job_pop_mtx.lock();
    if(known_jobs_[id].size() == 0)
        known_jobs_[id].reserve(for_param.num_threads_);
    known_jobs_[id].push_back(std::move(job_in));

    JobQueue j_q(id, for_param.thread_id_);
    work_queue_[thread_id].push(j_q);
    
    job_pop_mtx.unlock();

}


void ThreadPool::push_termination_job(int thread_id) {

    JobQueue j_q(-1, 0);
    work_queue_[thread_id].push(j_q);
}


void ThreadPool::run(int me) {
    thread_id_to_int_[std::this_thread::get_id()] = me;
    while(true) {
        
        job_pop_mtx.lock();
        if(work_queue_[me].size() != 0) {

            JobQueue j_q = work_queue_[me].front();
            work_queue_[me].pop();
            job_pop_mtx.unlock();
            
            int pragma_id = j_q.j_id_;
            int thread_id = j_q.thread_id_;

            if(pragma_id != 0) {
                if(pragma_id == -1)
                    break;
                
                job_pop_mtx.lock();
                std::vector<JobIn>::iterator j_itr;
                for(j_itr = known_jobs_[pragma_id].begin(); j_itr != known_jobs_[pragma_id].end(); ++ j_itr) {
                    if(j_itr->job_.for_param_.thread_id_ == thread_id)
                        break;
                }
               
                job_pop_mtx.unlock();
                ForParameter for_param = j_itr->job_.for_param_;
                
                try {
                    j_itr->job_.nested_base_->callme(for_param);
                }catch(std::exception& e){
                    //known_jobs_[pragma_id][thread_id].terminated_with_exceptions_ = true;
                    std::cerr << "Pragma " << pragma_id << " terminated with exception: " << e.what() << std::endl;
                }


                if(j_itr->job_type_.compare("OMPTaskDirective") == 0
                    || j_itr->job_type_.compare("OMPSingleDirective") == 0
                    || j_itr->job_type_.compare("OMPSectionsDirective") == 0)
                {
                    int barriers_number = sched_opt_[pragma_id].barriers_.size();
                    int barrier_id;
                    for(int i = 0; i < barriers_number; i ++) {
                        barrier_id = sched_opt_[pragma_id].barriers_[i];
                        join(barrier_id);
                    }
                }

                job_end.lock();
                j_itr->job_completed_ = true;
                j_itr->done_cond_var_->notify_one();
                job_end.unlock();
            }
        }else {
            job_pop_mtx.unlock();
        }
    }
}


void ThreadPool::join(Jobid_t job_id) {

    /*for(int i = 0; i < known_jobs_[job_id].size(); i ++) {
        if(known_jobs_[job_id][i].job_completed_ != true) {
            std::unique_lock<std::mutex> lk(cond_var_mtx);
            known_jobs_[job_id][i].done_cond_var_->wait(lk);
        }
    }*/
    //std::mutex cond_var_mtx;
    
    std::vector<JobIn>::iterator j_itr;
    for(j_itr = known_jobs_[job_id].begin(); j_itr != known_jobs_[job_id].end(); ++ j_itr) {
        job_end.lock();
        if((*j_itr).job_completed_ != true){
            job_end.unlock();
            std::unique_lock<std::mutex> lk(cond_var_mtx[std::this_thread::get_id()]);
            j_itr->done_cond_var_->wait(lk);            
        }else{
            job_end.unlock();
        }
    }
    job_pop_mtx.lock();
    known_jobs_.erase(job_id);
    job_pop_mtx.unlock();
}


void ThreadPool::joinall() {
    /* Push termination job in the working queue */
    std::cout << "Joinall" << std::endl;
    for (int i = 0; i < threads_pool_.size(); i ++)
        push_termination_job(i);
    
    /* Joining on all the threads in the thread pool */
    for(int i = 0; i < threads_pool_.size(); i++)
        threads_pool_[i].join();

}


int chartoint(const char *cc){
    std::string s(cc);
    char c;
    int n = 0;
    int tmp;
    int i = s.size();
    for(std::string::iterator sitr = s.begin(); sitr != s.end(); ++ sitr){
        c = *sitr;
        tmp = c - 48;
        tmp = tmp*pow(10, i-1);
        n += tmp;
        i --;
    }
    return n;
}

int chartoint(char *cc){
    const char *c = cc;
    return chartoint(c);
}
\end{lstlisting}

\chapter{Python}

\begin{lstlisting}[language=CCC, caption=graphCreator.py]
import sys
import pargraph as par
import copy
import schedule as sched
import profiler as pro
import time
import multiprocessing
import itertools
import random
import threading

""" Usage: call with <filename> <pragma_xml_file> <executable_name> <profiling_interations> True/False (for output) """

if __name__ == "__main__":

	pragma_xml = sys.argv[1]
	executable = sys.argv[2]
	count = int(sys.argv[3])
	output = sys.argv[4]
	execution_time = float(sys.argv[5])
	deadline = float(sys.argv[6])
	multi = sys.argv[7]

	#runs count time the executable and aggregates the informations in executable_profile.xml. The single profile outputs are saved as profile+iter.xml
	profile_xml = pro.profileCreator(count, executable)

	#return the nested dot graphs in code style (one for each function)
	visual_nested_graphs = par.getNesGraph(pragma_xml, profile_xml)

	#returns the graphs to be visualized and the object graphs in flow style (one for each function)
	(visual_flow_graphs, flow_graphs) = par.getParalGraph(pragma_xml, profile_xml) 

	i = 0

	for g in visual_nested_graphs:
		g.write_pdf('graphs/%s_code.pdf'%flow_graphs[i].type)
		g.write_dot('graphs/%s_code.dot'%flow_graphs[i].type)
		i += 1

	i = 0
	for g in visual_flow_graphs:
		g.write_pdf('graphs/%s_flow.pdf'%flow_graphs[i].type)
		g.write_dot('graphs/%s_flow.dot'%flow_graphs[i].type)
		i += 1

	#creates the flow type graph --> flow.xml
	par.dump_graphs(flow_graphs)
	#adding to the original xml the profiling informations --> code.xml
	pro.add_profile_xml(profile_xml, pragma_xml)
	#creating the total graph with the call-tree
	func_graph = par.create_complete_graph(visual_flow_graphs, profile_xml)
	#creating the graphs with the function calls
	func_graph.write_pdf('graphs/function_graphs.pdf')
	func_graph.write_dot('graphs/function_graphs.dot')

	#creating the expanded graph where the functions are inserted in the flow graph
	exp_flows = copy.deepcopy(flow_graphs)
	par.explode_graph(exp_flows)
	main_flow = sched.get_main(exp_flows)

	#creating a generator for the expanded graph
	gen = sched.generate_task(main_flow)

	#creating a new generator for the expanded graph
	sched.make_white(main_flow)

	#getting the number of physical cores of the machine profiled
	max_flows = sched.get_core_num(profile_xml)
	max_flows = 4
	#getting cores of the actual machine, but the problem is multithreading
	cores = multiprocessing.cpu_count()
	if cores == 1:
		cores = 2

	#initializing all the lists for the parallel scheduling algorithm
	tasks_list = []
	task_list = []
	flows_list = []
	optimal_flow_list = []
	p_list = []
	queue_list = []
	results = []
	num_tasks = 0

	#getting the number of tasks in the expanded graph and creating a list of task
	for task in gen:
		task_list.append(task)
		num_tasks += 1

	if output == 'True':
		sched.make_white(main_flow)
		par.scanGraph(main_flow)

	#starting the parallel or sequential search of the best solution with a timing constrain
	if multi == 'parallel':
		for core in range(cores):
			tmp = []
			optimal_flow_list.append(tmp)
			tmp_2 = []
			flows_list.append(tmp_2)
			random.shuffle(task_list)
			tasks_list.append(copy.deepcopy(task_list))
			q = sched.Queue()
			queue_list.append(q)
			p_list.append(multiprocessing.Process(target = sched.get_optimal_flow, args = (flows_list[core], tasks_list[core], 0, optimal_flow_list[core], num_tasks, max_flows, execution_time, queue_list[core],  )))
			print "starting core: ",core
			p_list[core].start()
		#getting the results from the processes
		for queue in queue_list:
			t = queue.q.get()
			results.append(t)
		#joining all the processes
		i = 0
		for p in p_list:
			p.join()
			print "core ", i, " joined"
			i += 1
		#getting the best result
		optimal_flow = results[0]
		best = 0
		for i in range(len(results)):
			print "result:"
			for flow in results[i]:
				flow.dump()
				if sched.get_cost(results[i]) < sched.get_cost(optimal_flow):
					best = i
		optimal_flow = results[best]
	else:
			optimal_flow = []
			flow_list = []
			execution_time += time.clock()
			print "searching best schedule"
			sched.get_optimal_flow_single(flow_list, task_list, 0, optimal_flow, num_tasks, max_flows, execution_time )



	#printing the best result	
	print "solution:"
	for flow in optimal_flow:
		flow.dump("\t")
		print "\ttime:",flow.time

	#substitutes "for tasks" with splitted versions if present in the optimal flows
	par.add_new_tasks(optimal_flow, main_flow)
	sched.make_white(main_flow)
	gen_ = sched.generate_task(main_flow)

	t_list = []
	for t in gen_:
		t_list.append(t)
		"""
		print t.type," @ ", t.start_line, " has parents:"
		for p in t.parent:
			print "\t ",p.type," @ ", p.start_line
		print "and children:"
		for c in t.children:
			print "\t ",c.type," @ ", c.start_line
		print
		"""

	#adds id's to all the tasks to retrive the flow to which they belong
	par.add_flow_id(optimal_flow, t_list)

	#sets arrival times and deadlines using a modified version of the chetto algorithm
	sched.chetto(main_flow, deadline, optimal_flow)

	#checks if the schedule is feasible and in case creates the schedule file
	if sched.check_schedule(main_flow):
		sched.create_schedule(main_flow, len(optimal_flow))
		sched.make_white(main_flow)
		#sched.print_schedule(main_flow)
	else:
		print "tasks not schedulable, try with more search time"

	#prints extended info of the entire pragma graph
\end{lstlisting}

\begin{lstlisting}[language=CCC, caption=pargraph.py]
import pydot as p
import profiler as pro
import xml.etree.cElementTree as ET
from random import randrange
import copy
import schedule as sched
import re
import math

colors = (	"beige",  "bisque3",	"bisque4",	"blanchedalmond",	   "blue",  
"blue1",	"blue2",	"blue3",	"blue4",	"blueviolet",
"brown",	"brown1",	"brown2",   "brown3", 	"brown4",
"burlywood",	"burlywood1",	"burlywood2",	"burlywood3",	"burlywood4",
"cadetblue",	"cadetblue1",	"cadetblue2",	"cadetblue3",	"cadetblue4",
"chartreuse",	"chartreuse1",	"chartreuse2",	"chartreuse3",	"chartreuse4",
"chocolate",	"chocolate1",	"chocolate2",	"chocolate3",	"chocolate4",
"coral",	"coral1",	"coral2",	"coral3",	"coral4",
"cornflowerblue",	"crimson",	   "cyan",   	"cyan1",	"cyan2",
"cyan3",	"cyan4",	"darkgoldenrod",	"darkgoldenrod1",	"darkgoldenrod2",
"darkgoldenrod3",	"darkgoldenrod4",	"darkgreen",	"darkkhaki",	"darkolivegreen",
"darkolivegreen1",	"darkolivegreen2",	"darkolivegreen3",	"darkolivegreen4",	"darkorange",
"darkorange1",	"darkorange2",	"darkorange3",	"darkorange4",	"darkorchid",
"darkorchid1",	"darkorchid2",	"darkorchid3",	"darkorchid4",	"darksalmon",
"darkseagreen",	"darkseagreen1",	"darkseagreen2",	"darkseagreen3",	"darkseagreen4",
"darkslateblue",	"darkslategray",	"darkslategray1",	"darkslategray2",	"darkslategray3",
"darkslategray4",	"darkslategrey",	"darkturquoise",	"darkviolet",	"deeppink",
"deeppink1",	"deeppink2",	"deeppink3",	"deeppink4",	"deepskyblue",
"deepskyblue1",	"deepskyblue2",	"deepskyblue3",	"deepskyblue4",	"dimgray",
"dimgrey",	"dodgerblue",	"dodgerblue1",	"dodgerblue2",	"dodgerblue3",
"dodgerblue4",	"firebrick",	"firebrick1",	"firebrick2",	"firebrick3",
"firebrick4",	"forestgreen", "gold",   	"gold1",	"gold2",	
"gold3",	"gold4",    "goldenrod",	"goldenrod1",	"goldenrod2",	"goldenrod3",	"goldenrod4")

class Node(object):
    def __init__(self, Ptype, s_line, time, variance):
        self.type = Ptype
        self.start_line = s_line
        self.children = []
        self.parent = []
        self.options = []
        self.time = float(time)
        self.variance = variance
        self.end_line = 0
        self.callerid = []
        self.deadline = None
        self.arrival = None
        self.d = None
        self.children_time = 0
        self.in_time = 0
        self.color = 'white'
        self.id = None
    def add(self, x):	
    	x.parent.append(self)
    	self.children.append(x)
    def myself(self):
		if self.type != 'BARRIER':
			print "pragma node: ", self.type, "\n    start_line: ", self.start_line, "\n    endl_line", self.end_line
			if self.type.find("_end") == -1:
				if self.time != 0:
					print "    time: ", self.time
					print "    variance: ", self.variance
					print "    children time: ", self.children_time
					print "    self time: ", self.in_time
				else:
					print "    not executed"
				if(len(self.options) != 0):
					print "    Options:"
					for i in self.options:
						print "        ",i[0]," ",i[1]
			print "     chetto deadline :", self.d
			print "     chetto arrival :", self.arrival
		else:
			print "pragma node: ", self.type, "\n    start_line: ", self.start_line
		print

class For_Node(Node): 
    def __init__(self, Ptype, s_line, init_type, init_var, init_value, init_cond, init_cond_value, init_increment, init_increment_value, time, variance, mean_loops):
        Node.__init__(self, Ptype, s_line, time, variance)
        self.init_type = init_type
        self.init_var = init_var
        self.init_value = init_value
        self.init_cond = init_cond
        self.init_cond_value = init_cond_value
        self.init_increment = init_increment
        self.init_increment_value = init_increment_value
        self.mean_loops = mean_loops
    def myself(self):
		print "for node: ", self.type, "\n    start_line: ", self.start_line, "\n    endl_line: ", self.end_line, "\n    init_type:", self.init_type, "\n    init_var: ", self.init_var, "\n    init_value: ", self.init_value,"\n    init_condition: ", self.init_cond, "\n    init_condition_value: ", self.init_cond_value, "\n    init_increment_type: ", self.init_increment,"\n    init_increment: ", self.init_increment_value, "\n    mean_loops:", self.mean_loops
		print "     chetto deadline :", self.d
		print "     chetto arrival :", self.arrival
		if(len(self.options) != 0):
			print "    Options:"
			for i in self.options:
				print "        ", i[0], " ", i[1]
		if self.time != 0:
			print "    time: ", self.time
			print "    variance: ", self.variance
			print "    children time: ", self.children_time,"\n"
			print "    self time: ", self.in_time,"\n"
		else:
			print "    not executed\n"

class Fx_Node(Node):
	def __init__(self, Ptype, line, returnType, time, variance, file_name):
		Node.__init__(self, Ptype, line, time, variance)
		self.arguments = []
		self.returnType = returnType
		self.time = float(time)
		self.file_name = file_name
	def add_arg(self, type_):
		self.arguments.append(type_)
	def myself(self):
		print "function node: ", self.type, "() {\n    line: ", self.start_line, "\n    return type: ", self.returnType
		print "     chetto deadline :", self.d
		print "     chetto arrival :", self.arrival
		if(len(self.arguments) != 0):
			print "    Parameters: "
			i = 0
			for par in self.arguments:
				print "   ", i, ") ", par[0], " ", par[1]
				i = i + 1
		else:
			print "    No input parameters"
		if self.time != 0:
			print "    time: ", self.in_time
			print "    variance: ", self.variance
			print "    children time: ", self.children_time,"\n}\n"
		else:
			print "    not executed\n}\n"

class Function():
	def __init__(self, time, variance, children_time):
		self.time = float(time)
		self.variance = variance
		self.pragmas = {}
		self.children_time = float(children_time)
		self.in_time = float(self.time) - float(self.children_time)
	def add_pragma(self, pragma):
		self.pragmas[pragma[0]] = (pragma[1], pragma[2], pragma[3], pragma[4], pragma[5])

class Architecture():
	def __init__(self, num_cores, tot_memory):
		self.num_cores = num_cores
		self.tot_memory = tot_memory

class Time_Node():
	def __init__(self, func_line, pragma_line ):
		self.times = []
		self.func_line = func_line
		self.pragma_line = pragma_line
		self.variance = 0
		self.loops = []
		self.caller_list = []
		self.children_time = []

class Flow():
	def __init__(self):
		self.tasks = []
		self.bandwidth = 0
		self.time = 0
	def add_task(self, task):
		self.tasks.append(task)
		self.update(task)
	def update(self, task):
		self.time += task.in_time #float(task.time) - float(task.children_time)
	def dump(self,prefix=""):
		print prefix,"flow:"
		for task in self.tasks:
			print prefix, "\t", task.type, " ", task.start_line, " ", task.in_time, " id ", task.id
	def remove_task(self, task):
		self.tasks.remove(task)
		self.time -= task.in_time #float(task.time) - float(task.children_time)

class Task():
	def __init__(self, count, id):
		self.count = count
		self.id = []
		self.id.append(id)


def scanGraph(node):
	#print pre, node.type
	if node.color != 'black':
		node.color = 'black'
		node.myself()
		print "     has children:"
		for c in node.children:
			print "          ",c.type,"@",c.start_line
		print "     has parent:"
		for p in node.parent :
			print "          ",p.type,"@",p.start_line
		for n in node.children:
			scanGraph(n)

def indent(elem, level=0):
    i = "\n" + level * "  "
    if len(elem):
        if not elem.text or not elem.text.strip():
            elem.text = i + "  "
        if not elem.tail or not elem.tail.strip():
            elem.tail = i
        for elem in elem:
            indent(elem, level + 1)
        if not elem.tail or not elem.tail.strip():
            elem.tail = i
    else:
        if level and (not elem.tail or not elem.tail.strip()):
            elem.tail = i

def getParalGraph(pragma_xml, profile_xml):
	pragma_graph_root = ET.ElementTree(file = pragma_xml).getroot()
	profile_graph_root = ET.ElementTree(file = profile_xml).getroot()

	functions = pro.getProfilesMap(profile_xml)
	objGraph = []
	graphs = []
	count = 0
	arch = Architecture(profile_graph_root.find('Hardware/NumberofCores').text, profile_graph_root.find('Hardware/MemorySize').text)

	file_name = pragma_graph_root.find('Name').text

	for n in pragma_graph_root.findall('Function'):
		graphs.append(p.Dot(graph_type = 'digraph'))
		name = n.find('Name').text
		time = float(functions[n.find('Line').text].time)
		callerid = functions[n.find('Line').text].callerid
		children_time = float(functions[n.find('Line').text].children_time)
		root = n.find('Line').text
		if (time == 0):
			pragma_graph_root = p.Node(n.find('Line').text, label = name + "()\nnot executed", root = root)
		else:
			pragma_graph_root = p.Node(n.find('Line').text, label = name + "()\nexecution time %g" % time, root = root)
		pragma_graph_root.callerid = callerid
		graphs[count].add_node(pragma_graph_root)
		Objroot = Fx_Node(name, n.find('Line').text,n.find('ReturnType').text, float(functions[n.find('Line').text].time), functions[n.find('Line').text].variance, file_name)
		for par in n.findall('Parameters/Parameter'):
			Objroot.add_arg( ( par.find('Type').text,par.find('Name').text ) )
		Objroot.children_time = children_time
		Objroot.in_time = Objroot.time - children_time
		for caller in functions[n.find('Line').text].callerid:
			Objroot.callerid.append(caller)
		objGraph.append(Objroot)
		scan(n, graphs[count], pragma_graph_root, objGraph[count], functions[n.find('Line').text].pragmas, root)
		count = count + 1
	return (graphs, objGraph)

def scan(xml_tree, pragma_graph, node, treeNode, func_pragmas, root):
	for d in xml_tree.find('Pragmas').findall('Pragma'):
		end_line = d.find('Position/EndLine').text
		key = d.find('Position/StartLine').text

		if key not in func_pragmas:
			time = 0
			variance = None
			loops = 0
			callerid = None
			children_time = 0
		else:
			time = float(func_pragmas[key][0])
			variance = func_pragmas[key][1]
			loops = func_pragmas[key][2]
			callerid = func_pragmas[key][3]
			children_time =  float(func_pragmas[key][4])

		tmp_name = d.find('Name').text.replace("::", " ")
		visual_name = tmp_name+"@%s"%key

		if ("For" in tmp_name ):
			if (d.find('For/Declaration/InitValue') != None):
				init_value = d.find('For/Declaration/InitValue').text
			else:
				init_value = d.find('For/Declaration/InitVariable').text
			if (d.find('For/Condition/ConditionValue') != None):
				init_var = d.find('For/Condition/ConditionValue').text
			else:
				init_var = d.find('For/Condition/ConditionVariable').text
			if(d.find('For/Increment/IncrementValue') != None):
				inc = d.find('For/Increment/IncrementValue').text
			else:
				inc = ""
			Objchild = For_Node(tmp_name, d.find('Position/StartLine').text, d.find('For/Declaration/Type').text, d.find('For/Declaration/LoopVariable').text, init_value, d.find('For/Condition/Op').text, init_var, d.find('For/Increment/Op').text, inc, time, variance, loops )
			visual_name = visual_name + "\nfor( " + Objchild.init_var + " = " + Objchild.init_value + "; " + Objchild.init_var + " " + Objchild.init_cond + " " + Objchild.init_cond_value + "; " + Objchild.init_var + " " + Objchild.init_increment + " " + Objchild.init_increment_value + ")"
		else:
			Objchild = Node(tmp_name, key, time, variance )

		deadline = None
		if(d.find('Options')):
			for op in d.findall('Options/Option'):
				Objchild.options.append( (op.find('Name').text,[get_parameter(i) for i in op.findall('Parameter')]) )
				if op.find('Name').text == 'deadline':
					deadline = op.find('Parameter').text
		Objchild.end_line = end_line
		Objchild.callerid.append(callerid)
		Objchild.deadline = deadline
		Objchild.children_time = children_time
		Objchild.in_time = Objchild.time - children_time
		if (time == 0):
			child = p.Node(key, label = visual_name + "\nnot executed", root = root)
		else:
			child = p.Node(key, label = visual_name + "\nexecution time: " + str(time) + "\nvariance: " + str(variance), root = root)
		pragma_graph.add_node(node)
		pragma_graph.add_node(child)
		pragma_graph.add_edge(p.Edge(node, child))
		treeNode.add(Objchild)
		#print Objchild.type,"@",Objchild.start_line," is attached to ",treeNode.type,"@",treeNode.start_line

		if(d.find('Children')):
			node_ = create_diamond(d.find('Children'), pragma_graph, child, Objchild, func_pragmas, root)
			tmp_name = (node_.start_line)
			if tmp_name not in func_pragmas:
				time = 0
			else:
				time = func_pragmas[tmp_name][0]
			#treeNode = Node('BARRIER_end', tmp_name, 0, 0)
			#Objchild.add(treeNode)
			treeNode = node_
			node = p.Node(tmp_name + "_end", label = "BARRIER", root = root)
		else:
			node = child
			treeNode = Objchild

def create_diamond(tree, graph, node, treeNode, func_pragmas, root):
	special_node = p.Node(node.get_name().replace("\"", "") + "_end", label = 'BARRIER', root = root)
	Objspecial_node = Node( 'BARRIER_end' , node.get_name() , 0, 0 )
	color = colors[randrange(len(colors) - 1)]
	for d in tree.find('Pragmas').findall('Pragma'):

		end_line = d.find('Position/EndLine').text
		key = d.find('Position/StartLine').text

		if key not in func_pragmas:
			time = 0
			variance = None
			loops = 0
			callerid = None
			children_time = 0
		else:
			time = float(func_pragmas[key][0])
			variance = func_pragmas[key][1]
			loops = func_pragmas[key][2]
			callerid = func_pragmas[key][3]
			children_time = float(func_pragmas[key][4])

		tmp_name = d.find('Name').text.replace("::", " ")
		visual_name = tmp_name + "@%s" % key

		if ("For" in tmp_name ):
			loops = func_pragmas[key][2]
			if (d.find('For/Declaration/InitValue') != None):
				init_value = d.find('For/Declaration/InitValue').text
			else:
				init_value = d.find('For/Declaration/InitVariable').text
			if (d.find('For/Condition/ConditionValue') != None):
				init_var = d.find('For/Condition/ConditionValue').text
			else:
				init_var = d.find('For/Condition/ConditionVariable').text
			if(d.find('For/Increment/IncrementValue') != None):
				inc = d.find('For/Increment/IncrementValue').text
			else:
				inc = ""
			Objchild = For_Node(tmp_name, key, d.find('For/Declaration/Type').text, d.find('For/Declaration/LoopVariable').text, init_value, d.find('For/Condition/Op').text, init_var, d.find('For/Increment/Op').text, inc , time, variance, loops)
			visual_name = visual_name + "\nfor( " + Objchild.init_var + " = " + Objchild.init_value + "; "+Objchild.init_var + " " + Objchild.init_cond + " " + Objchild.init_cond_value + "; " + Objchild.init_var + " " + Objchild.init_increment + " " + Objchild.init_increment_value + ")"
		else:
			Objchild = Node(tmp_name, key, time, variance)

		deadline = None
		if(d.find('Options')):
			for op in d.find('Options').findall('Option'):
				Objchild.options.append( (op.find('Name').text,[get_parameter(i) for i in op.findall('Parameter')]) )
				if op.find('Name').text == 'deadline':
					deadline = op.find('Parameter').text

		Objchild.end_line = end_line
		Objchild.callerid.append(callerid)
		Objchild.deadline = deadline
		Objchild.children_time = children_time
		Objchild.in_time = Objchild.time - children_time

		child = p.Node(key, label = visual_name + "\nexecution time: " + str(time) + "\nvariance: " + str(variance), root = root)
		graph.add_node(node)
		graph.add_node(child)
		graph.add_edge(p.Edge(node, child, color = color))
		treeNode.add(Objchild)

		if(d.find('Children')):
			#get the real returned label as name
			tmp_node = create_diamond(d.find('Children'), graph, child, Objchild, func_pragmas, root)
			g_node = p.Node(tmp_node.start_line+ "_end", label = 'BARRIER', root = root)
			graph.add_node(g_node)
			graph.add_node(special_node)
			graph.add_edge(p.Edge(g_node, special_node, color = color))
			#tmp_name = tmp.get_name().replace("\"", "") 
			#ObjTmp = Node(tmp_name, tmp_name, 0, 0)
			tmp_node.add(Objspecial_node)
		else:
			graph.add_node(child)
			graph.add_node(special_node)
			graph.add_edge(p.Edge(child, special_node, color = color))
			Objchild.add(Objspecial_node)
	return Objspecial_node

def find_nesting(tree, graph, node, func_pragmas, pre = ""):
	color = colors[randrange(len(colors) - 1)]
	for d in tree.find('Pragmas').findall('Pragma'):
		key = d.find('Position/StartLine').text
		if(key in func_pragmas):
			time = "\n execution time: " + str(func_pragmas[key][0])
			variance = "\nvariance: " + str(func_pragmas[key][1])
		else:
			time = "\nnot executed"
			variance = ""
		name = d.find('Name').text.replace("::"," ") + "@%s" % key
		child = p.Node(name, label = name + time + variance)
		graph.add_node(node)
		graph.add_node(child)
		graph.add_edge(p.Edge(node, child, color = color ))
		#print pre+name
		if(d.find('Children')):
			find_nesting(d.find('Children'), graph, child, func_pragmas, pre + " ")

def getNesGraph(xml, profile_xml):
	tree = ET.ElementTree(file = xml) 
	profile_graph_root = ET.ElementTree(file = profile_xml).getroot()
	functions = pro.getProfilesMap(profile_xml)

	root = tree.getroot()
	graphs = []
	count = 0

	for n in root.iter('Function'):
		key = n.find('Line').text
		time = float(functions[key].time)
		variance = functions[key].variance
		graphs.append(p.Dot(graph_type = 'digraph'))
		name = n.find('Name').text
		if (time == 0):
			root = p.Node(name, label = name + "()" + "\nnot executed")
		else:
			root = p.Node(name, label = name + "()" + "\n execution time: %f" % time + "\nvariance: " + str(variance))
		graphs[count].add_node(root)
		find_nesting(n, graphs[count], root, functions[key].pragmas)
		count += 1

	return graphs

def create_complete_graph(visual_flow_graphs, profile_xml):
	func_graph = p.Dot(graph_type = 'digraph', compound = 'true')
	clusters = []

	i = 0

	for func in visual_flow_graphs:
		clusters.append(p.Cluster(str(i)))
		for node in func.get_nodes():
			clusters[i].add_node(node)
		for edge in func.get_edge_list():
			clusters[i].add_edge(edge)
		func_graph.add_subgraph(clusters[i])
		i +=  1

	functions_callers = pro.get_table(profile_xml)

	for func in visual_flow_graphs:
		root = func.get_nodes()[0].obj_dict['attributes']['root']
		if len(functions_callers[root]) > 0 :
			for caller in functions_callers[root]:
				func_graph.add_edge(p.Edge(caller, root))

	return func_graph

def dump_graphs(flow_graphs):
	root = ET.Element('File')
	name = ET.SubElement(root, 'Name')
	name.text = flow_graphs[0].file_name
	graph_type = ET.SubElement(root, 'GraphType')
	graph_type.text = "flow"
	for func in flow_graphs:
		function = ET.SubElement(root, 'Function')
		function.attrib['id'] = str(func.start_line) + str(func.end_line)
		func_name = ET.SubElement(function, 'Name')
		func_name.text = func.type
		returnType = ET.SubElement(function, 'ReturnType')
		returnType.text = func.returnType
		if len(func.arguments) != 0:
			parameters = ET.SubElement(function, 'Parameters')
			for par in func.arguments:
				parameter = ET.SubElement( parameters, 'Parameter')
				type_ = ET.SubElement( parameter, 'Type')
				type_.text = par[0]
				name_ = ET.SubElement( parameter, 'Name')
				name_.text = par[1]
		line = ET.SubElement(function, 'Line')
		line.text = func.start_line 
		time = ET.SubElement(function, 'Time')
		time.text = str(func.time)
		variance = ET.SubElement(function, 'Variance')
		variance.text = str(func.variance)
		func.xml_parent = None
		if ( func.callerid != None ):
			callerids = ET.SubElement(function, 'Callerids')
			for id_ in func.callerid:
				callerid = ET.SubElement(callerids, 'Callerid')
				callerid.text = id_
		if len(func.children) != 0:
			pragma_list = []
			edge_list = []
			pragmas = ET.SubElement(function, 'Nodes')
			dump_pragmas(func, pragmas, pragma_list)
			edges = ET.SubElement(function, 'Edges')
			dump_edges(func, edges, edge_list)

	tree = ET.ElementTree(root)
	indent(tree.getroot())
	tree.write('flow.xml')

def dump_pragmas(pragma_node, pragmas_element, pragma_list):
	for pragma in pragma_node.children:
		if str(pragma.start_line) + str(pragma.end_line) not in pragma_list:
			pragma_list.append(str(pragma.start_line) + str(pragma.end_line))
			pragma_ = ET.SubElement(pragmas_element, 'Pragma')
			pragma_.attrib['id'] = str(pragma.start_line) + str(pragma.end_line)
			name = ET.SubElement(pragma_, 'Name')
			if not "_end" in pragma.type:
				name.text = pragma.type
			else:
				name.text = "BARRIER"
			if(len(pragma.options) != 0):
				options = ET.SubElement(pragma_, 'Options')
				for op in pragma.options:
					option = ET.SubElement(options, 'Option')
					op_name = ET.SubElement(option, 'Name')
					op_name.text = op[0]
					for par in op[1]:
						op_parameter = ET.SubElement(option, 'Parameter')
						op_var = ET.SubElement(op_parameter, 'Var')
						op_var.text = par[1]
						op_type = ET.SubElement(op_parameter, 'Type')
						op_type.text = par[0]
			position = ET.SubElement(pragma_, 'Position')
			start = ET.SubElement(position, 'StartLine')
			start.text = pragma.start_line
			if(name.text != "BARRIER"):
				end = ET.SubElement(position, 'EndLine')
				end.text = pragma.end_line
			if (pragma.callerid != None ):
				callerids = ET.SubElement(pragma_, 'Callerids')
				for id_ in pragma.callerid:
					callerid = ET.SubElement(callerids, 'Callerid')
					callerid.text = id_
			if(pragma.time != 0):
				time = ET.SubElement(pragma_, 'Time')
				time.text = str(pragma.time)
			if(pragma.variance != None):
				variance = ET.SubElement(pragma_, 'Variance')
				variance.text = str(pragma.variance)

		dump_pragmas(pragma, pragmas_element, pragma_list)

def dump_edges(pragma_node, edges_element, pragma_list):
	for pragma in pragma_node.children:
		if pragma_node.start_line + pragma.start_line not in pragma_list:
			pragma_list.append(pragma_node.start_line+pragma.start_line)
			edge = ET.SubElement(edges_element, 'Edge')
			source = ET.SubElement(edge, 'Source')
			source.text = str(pragma_node.start_line) + str(pragma_node.end_line)
			dest = ET.SubElement(edge, 'Dest')
			dest.text = str(pragma.start_line) + str(pragma.end_line)
		dump_edges(pragma,edges_element, pragma_list)

def find_node(node, flow_graphs):
	for function in flow_graphs:
		tmp_node = find_sub_node(node, function) 
		if tmp_node != None :
			return tmp_node

def find_node2(key_start, key_parent, flow_graphs):
	tmp_node = find_sub_node2(key_start, key_parent, flow_graphs) 
	if tmp_node != None :
		return tmp_node

def find_sub_node2(key_start, key_parent, function):
	if (function.start_line) == key_start and ('BARRIER' not in function.type):
			return function
	for child in function.children:
		if (child.start_line) == key_start and ('BARRIER' not in child.type) and child.parent[0].start_line == key_parent:
			return child
		else:
			tmp_node = find_sub_node2(key_start, key_parent, child)
		if tmp_node != None:
			return tmp_node
	return None

def find_sub_node(node, function):
	if (function.start_line) == node and ('BARRIER' not in function.type):
			return function
	for child in function.children:
		if (child.start_line) == node and ('BARRIER' not in child.type):
			return child
		else:
			tmp_node = find_sub_node(node, child)
		if tmp_node != None:
			return tmp_node
	return None

class Caller():
	def __init__(self, original_caller, used_caller):
		self.original_caller = original_caller
		self.used_caller = used_caller
		self.old_children = []

#adding to the main graph all the function which are called taking care of multiple connections between pragma and caller
def explode_graph(flow_graphs):
	setted_callers = {}
	for function in flow_graphs:
		count = 0
		caller_list = function.callerid
		if caller_list != None:
			for caller in caller_list:
				function_copy = copy.deepcopy(function)
				count += 1
				caller_node = find_node(caller, flow_graphs)
				if caller_node.start_line not in setted_callers:
					setted_callers[caller_node.start_line] = Caller(copy.copy(caller_node), caller_node)
					function_copy.parent.append(caller_node)
					children_list = []
					for child in caller_node.children:
						children_list.append(child)
						child.parent.remove(caller_node)
						setted_callers[caller_node.start_line].old_children.append(child)
					caller_node.children = []
					caller_node.children.append(function_copy)
					last_node = sched.get_last(function_copy)
					last_node.children = children_list
					for child in children_list:
						child.parent.append(last_node)
				else:
					children_list = []
					for child in setted_callers[caller_node.start_line].old_children:
						children_list.append(child)
					function_copy.parent.append(setted_callers[caller_node.start_line].used_caller)
					setted_callers[caller_node.start_line].used_caller.children.append(function_copy)
					last_node = sched.get_last(function_copy)
					last_node.children = children_list
					for child in children_list:
						child.parent.append(last_node)


def get_parameter(parameter):
	if parameter.find('Type') != None:
		type_ = parameter.find('Type').text
	else:
		type_ = 'None'
	return (type_, parameter.find('Var').text)

def create_map(optimal_flow):
	for_map = {}
	for flow in optimal_flow:
		for task in flow.tasks:
			if "splitted" in task.type:
				l = re.findall(r'\d+',task.type)
				id = str(l[0]) + "_" + str(l[2])
				if id in for_map:
					for_map[id].count += 1
					for_map[id].id.append(task.id)
				else:
					for_map[id] = Task(1, task.id)
	return for_map

def add_new_tasks(optimal_flow, main_flow):
	for_map = create_map(optimal_flow)
	for key in for_map:
		l = re.findall(r'\d+',key)
		node_to_replace = find_node2(l[0], l[1], main_flow)
		nodes_to_add = []

		for i in range(for_map[key].count):
			nodes_to_add.append(For_Node("splitted_" + node_to_replace.start_line + "." + str(i), node_to_replace.start_line, node_to_replace.init_type, node_to_replace.init_var, node_to_replace.init_value, node_to_replace.init_cond, node_to_replace.init_cond_value, node_to_replace.init_increment, node_to_replace.init_increment_value, node_to_replace.time, node_to_replace.variance, math.floor(float(node_to_replace.mean_loops) / (i + 1))))

		for parent in node_to_replace.parent:
			parent.children.remove(node_to_replace)
			for n in nodes_to_add:
				parent.add(n)
				n.id = for_map[key].id.pop(0)
				n.color = 'white'
				n.from_type = node_to_replace.type

		for child in node_to_replace.children:
			child.parent.remove(node_to_replace)
			for n in nodes_to_add:
				n.add(child) 


def add_flow_id(optimal_flow, task_list):
	id_map = {}
	for flow in optimal_flow:
		for task in flow.tasks:
			if "splitted" not in task.type:
				if task.start_line not in id_map:
					id_map[task.start_line] = task.id
				else:
					id_map[task.start_line + str(1)] = task.id
	for task in task_list:
		if "splitted" not in task.type:
			if task.start_line in id_map:
				task.id = id_map[task.start_line]
				id_map.pop(task.start_line, None)
			else:
				task.id = id_map[task.start_line + str(1)]

\end{lstlisting}

\begin{lstlisting}[language=CCC, caption=profiler.py]
from __future__ import with_statement
import os
import pargraph as par
import xml.etree.cElementTree as ET
import numpy 
import re

def profileCreator(cycle, executable):
	pragma_times = {}
	function_times = {}
	j = 0
	param_string = ''

	if os.path.exists("./parameters.txt"):
		with open("./parameters.txt","r") as f:
			parameters = f.readlines()
		for s in parameters:
			param_string +=  s.strip()

	for i in range(cycle):
		print "profiling iteration: " + str((j + 1))
		os.system("./" + executable + " " + param_string + " >/dev/null")	
		os.system("mv log_file.xml " + "./logfile%s.xml" % j)
		root = ET.ElementTree(file = "./logfile%s.xml" % j).getroot()

		for pragma in root.iter('Pragma'):
			key = pragma.attrib['fid'] + pragma.attrib['pid']
			if (key not in pragma_times):
				pragma_times[key] = par.Time_Node(int(pragma.attrib['fid']), int(pragma.attrib['pid']))
			if ('callerid' in pragma.attrib):
				if pragma.attrib['callerid'] not in pragma_times[key].caller_list:
					pragma_times[key].caller_list.append(pragma.attrib['callerid'])
			if ('loops' in pragma.attrib):
				pragma_times[key].loops.append(int(pragma.attrib['loops']))
			if ('time' in pragma.attrib):
				pragma_times[key].time = pragma.attrib['time']
			if ('childrenTime' in pragma.attrib):
				pragma_times[key].children_time.append(float(pragma.attrib['childrenTime']))
			pragma_times[key].times.append(float(pragma.attrib['elapsedTime']))

		for func in root.iter('Function'):
			key = func.attrib['fid']
			if (key in function_times):
				function_times[key].times.append(float(func.attrib['elapsedTime']))
			else:
				function_times[key] = par.Time_Node(int(func.attrib['fid']), 0)
				function_times[key].times.append(float(func.attrib['elapsedTime']))
			if ('callerid' in func.attrib):
				if int(func.attrib['callerid']) not in function_times[key].caller_list:
					function_times[key].caller_list.append(int(func.attrib['callerid']))
			if ('time' in func.attrib):
				function_times[key].time = func.attrib['time']
			if ('childrenTime' in func.attrib):
				function_times[key].children_time.append(float(func.attrib['childrenTime']))

		j += 1

	num_cores = ET.ElementTree(file = "logfile0.xml").getroot().find('Hardware').attrib['NumberofCores']
	tot_memory = ET.ElementTree(file = "logfile0.xml").getroot().find('Hardware').attrib['MemorySize']

	root = ET.Element('Log_file')
	h = ET.SubElement(root, 'Hardware')
	h1 = ET.SubElement(h, 'NumberofCores')
	h2 = ET.SubElement(h, 'MemorySize')
	h1.text = num_cores
	h2.text = tot_memory

	for key in function_times:
		s = ET.SubElement(root, 'Function')
		line = ET.SubElement(s, 'FunctionLine')
		time = ET.SubElement(s, 'Time')
		var = ET.SubElement(s, 'Variance')
		if (len(function_times[key].caller_list) != 0 ):
			callerid = ET.SubElement(s, 'CallerId')
			callerid.text = str(function_times[key].caller_list)
		if (len(function_times[key].children_time) != 0):
			children_time = ET.SubElement(s, 'ChildrenTime')
			children_time.text = str(numpy.mean(function_times[key].children_time))
		time.text = str(numpy.mean(function_times[key].times))
		line.text = str(function_times[key].func_line)
		var.text = str(numpy.std(function_times[key].times))

	for key in pragma_times:
		s = ET.SubElement(root, 'Pragma')
		f_line = ET.SubElement(s, 'FunctionLine')
		p_line = ET.SubElement(s, 'PragmaLine')
		time = ET.SubElement(s, 'Time')
		var = ET.SubElement(s, 'Variance')
		if (len(pragma_times[key].loops) != 0):
			loops = ET.SubElement(s, 'Loops')
			loops.text = str(numpy.mean(pragma_times[key].loops))
		if (len(pragma_times[key].caller_list) != 0 ):
			callerid = ET.SubElement(s, 'CallerId')
			callerid.text = str(pragma_times[key].caller_list)
		if (len(pragma_times[key].children_time) != 0):
			children_time = ET.SubElement(s, 'ChildrenTime')
			children_time.text = str(numpy.mean(pragma_times[key].children_time))
		time.text = str(numpy.mean(pragma_times[key].times))
		f_line.text = str(pragma_times[key].func_line)
		p_line.text = str(pragma_times[key].pragma_line)
		var.text = str(numpy.std(pragma_times[key].times))

	tree = ET.ElementTree(root)
	par.indent(tree.getroot())
	tree.write(executable + "_profile.xml")

	return executable + "_profile.xml"

def add_profile_xml(profile_xml, xml_tree):
	functions = getProfilesMap(profile_xml)
	tree = ET.ElementTree(file = xml_tree) 
	root = tree.getroot()
	type_ = ET.SubElement(root, 'GraphType')
	type_.text = 'Code'

	for func in root.findall('Function'):
		key = func.find('Line').text
		func_time =  ET.SubElement(func, 'Time')
		func_time.text = str(functions[key].time)
		func_variance = ET.SubElement(func, 'Variance')
		func_variance.text = str(functions[key].variance)
		if len(functions[key].callerid) > 0:
			func_caller_ids = ET.SubElement(func, 'Callerids')
			tmp_list = set(functions[key].callerid)
			for id in tmp_list:
				func_caller_id = ET.SubElement(func_caller_ids,'Callerid')
				func_caller_id.text = id
		for pragma in func.iter('Pragma'):
			pragma_key = pragma.find('Position/StartLine').text
			if pragma_key in functions[key].pragmas:
				pragma_time = ET.SubElement(pragma, 'Time')
				pragma_time.text = functions[key].pragmas[pragma_key][0]
				pragma_variance = ET.SubElement(pragma, 'Variance')
				pragma_variance.text = functions[key].pragmas[pragma_key][1]
				if (functions[key].pragmas[pragma_key][2] != 0):
					pragma_loops = ET.SubElement(pragma, 'Loops')
					pragma_loops.text = functions[key].pragmas[pragma_key][2]	
				if 	(functions[key].pragmas[pragma_key][3] != None):
					pragma_callerid = ET.SubElement(pragma, 'Callerid')
					pragma_callerid.text = functions[key].pragmas[pragma_key][3].replace('[','').replace(']','').replace('\'','')

	par.indent(tree.getroot())			
	tree.write('code.xml')

def get_table(profile_xml):
	tree = ET.ElementTree(file = profile_xml) 
	root = tree.getroot()
	table = {}

	for func in root.iter('Function'):
		table[func.find('FunctionLine').text] = []
		if func.find('CallerId') != None:
			l = re.findall(r'\d+',func.find('CallerId').text)
			for j in l:
				table[func.find('FunctionLine').text].append(j)

	return table

def getProfilesMap(profile_xml):
	profile_graph_root = ET.ElementTree(file = profile_xml).getroot()

	functions = {}
	l = []

	for func in profile_graph_root.findall('Function'):
		f = par.Function(func.find('Time').text, func.find('Variance').text, func.find('ChildrenTime').text)
		f.callerid = []
		if (func.find('CallerId') != None):
			l = re.findall(r'\d+',func.find('CallerId').text.replace("[","").replace("]",""))
			for id_ in l:
				f.callerid.append(id_)
		functions[func.find('FunctionLine').text] = f

	for pragma in profile_graph_root.findall('Pragma'):
		if pragma.find('CallerId') != None:
			callerid = pragma.find('CallerId').text.replace("[\'","").replace("\']","")
		else:
			callerid = None
		if (pragma.find('Loops') != None):
			loops = pragma.find('Loops').text
		else :
			loops = 0
		functions[pragma.find('FunctionLine').text].add_pragma( (pragma.find('PragmaLine').text, pragma.find('Time').text, pragma.find('Variance').text, loops, callerid, pragma.find('ChildrenTime').text ))

	return functions
\end{lstlisting}

\begin{lstlisting}[language=CCC, caption=graphCreator.py]
import pargraph as par
import xml.etree.cElementTree as ET
import math
import copy
import time
import multiprocessing

class Queue():
	def __init__(self):
		self.q = multiprocessing.Queue()
		self.set = False

#returns the optimal flows 
#if time is to big for the number of possible solutions it does not work.

def get_optimal_flow(flow_list, task_list, level, optimal_flow, NUM_TASKS, MAX_FLOWS, execution_time, q):
	if time.clock() < execution_time :
		curopt = get_cost(optimal_flow)
		cur = get_cost(flow_list)
		if len(flow_list) < MAX_FLOWS and len(task_list) != level and cur <= curopt :
			task_i = task_list[level]
			# test integrating the single task in each
			for flow in flow_list :
				flow.add_task(task_i)
				get_optimal_flow(flow_list, task_list, level + 1, optimal_flow, NUM_TASKS, MAX_FLOWS, execution_time, q)
				flow.remove_task(task_i)
			new_flow = par.Flow()		
			new_flow.add_task(task_i)
			flow_list.append(new_flow)
			get_optimal_flow(flow_list, task_list, level + 1, optimal_flow, NUM_TASKS, MAX_FLOWS, execution_time, q)
			flow_list.remove(new_flow)

			if 'For' in task_i.type :
				#checks the possible splittings of the for node
				for i in range(2, MAX_FLOWS + 1):
					tmp_task_list = []
					#splits the for node in j nodes
					for j in range(0, i):
						task = par.For_Node("splitted_" + task_i.start_line + "." + str(j) + "_" + task_i.parent[0].start_line, task_i.start_line, task_i.init_type, task_i.init_var, task_i.init_value, task_i.init_cond, task_i.init_cond_value, task_i.init_increment, task_i.init_increment_value, task_i.time, task_i.variance, math.floor(float(task_i.mean_loops) / i))
						task.in_time = float(task_i.time) / i
						task_list.append(task)
						tmp_task_list.append(task)
					get_optimal_flow(flow_list, task_list, level + 1, optimal_flow, NUM_TASKS + i - 1, MAX_FLOWS, execution_time, q)
					for tmp_task in tmp_task_list:
						task_list.remove(tmp_task)		
		else:
			if len(task_list) == level and len(flow_list) == MAX_FLOWS and cur <= curopt:
				if cur < curopt or (get_num_splitted(flow_list) > get_num_splitted(optimal_flow) and get_num_splitted(flow_list) < (MAX_FLOWS * 2)):
					#print "acutal cost: ", get_cost(flow_list), "optimal cost: ", get_cost(optimal_flow)
					del optimal_flow[:]
					id = 0
					#print "newflowset:"
					for flow in flow_list:
						for task in flow.tasks:
							task.id = id
						id += 1
						optimal_flow.append(copy.deepcopy(flow))
					while( not q.q.empty() ):
						q.q.get()
					q.q.put(optimal_flow)

def get_optimal_flow_single(flow_list, task_list, level, optimal_flow, NUM_TASKS, MAX_FLOWS, execution_time):
	#print "time: ", time.clock() - execution_time
	if time.clock() < execution_time :
		curopt = get_cost(optimal_flow)
		cur = get_cost(flow_list)
		if len(flow_list) < MAX_FLOWS and len(task_list) != level and cur <= curopt :
			task_i = task_list[level]
			# test integrating the single task in each
			for flow in flow_list :
				flow.add_task(task_i)
				get_optimal_flow_single(flow_list, task_list, level + 1, optimal_flow, NUM_TASKS, MAX_FLOWS, execution_time)
				flow.remove_task(task_i)
			new_flow = par.Flow()		
			new_flow.add_task(task_i)
			flow_list.append(new_flow)
			get_optimal_flow_single(flow_list, task_list, level + 1, optimal_flow, NUM_TASKS, MAX_FLOWS, execution_time)
			flow_list.remove(new_flow)

			if 'For' in task_i.type :
				#checks the possible splittings of the for node
				for i in range(2, MAX_FLOWS + 1):
					tmp_task_list = []
					#splits the for node in j nodes
					for j in range(0, i):
						task = par.For_Node("splitted_" + task_i.start_line + "." + str(j) + "_" + task_i.parent[0].start_line, task_i.start_line, task_i.init_type, task_i.init_var, task_i.init_value, task_i.init_cond, task_i.init_cond_value, task_i.init_increment, task_i.init_increment_value, task_i.time, task_i.variance, math.floor(float(task_i.mean_loops) / i))
						task.in_time = float(task_i.time) / i
						task_list.append(task)
						tmp_task_list.append(task)
					get_optimal_flow_single(flow_list, task_list, level + 1, optimal_flow, NUM_TASKS + i - 1, MAX_FLOWS, execution_time)
					for tmp_task in tmp_task_list:
						task_list.remove(tmp_task)		
		else:
			if len(task_list) == level and len(flow_list) == MAX_FLOWS and cur <= curopt:
				if cur < curopt and get_num_splitted(flow_list) <  MAX_FLOWS/2 or (get_num_splitted(flow_list) > get_num_splitted(optimal_flow) and get_num_splitted(flow_list) <  MAX_FLOWS/2) :
					#print "acutal cost: ", get_cost(flow_list), "optimal cost: ", get_cost(optimal_flow)
					del optimal_flow[:]
					id = 0
					#print "newflowset:"
					for flow in flow_list:
						for task in flow.tasks:
							task.id = id
						id += 1
						optimal_flow.append(copy.deepcopy(flow))

def get_num_splitted(flow_list):
	num = 0
	for flow in flow_list:
		for task in flow.tasks:
			if 'splitted' in task.type:
				num += 1
	return num

#generator for the tasks of the graph
def generate_task(node):
	if node.color == 'white':
		node.color = 'black'
		yield node
		for n in node.children:
			for node in generate_task(n):
				yield node

def generate_list(l, node):
	if node.color == 'white':
		node.color = 'black'
		l.append(node)
		for n in node.children:
			generate_list(l, n)

#returns the number or physical cores
def get_core_num(profile):
	root = ET.ElementTree(file = profile).getroot()
	return int(root.find('Hardware/NumberofCores').text)

#sets the color of each node to white
def make_white(node):
	if node.color == 'black':
		node.color = 'white'
	for child in node.children:
		make_white(child)

#returns the graph which contains the 'main' function
def get_main(exp_flows):
	for i in range(len(exp_flows)):
		if exp_flows[i].type == 'main':
			return exp_flows[i]

#returns the last node of the input graph
def get_last(node):
	if not node.children:
		return node	
	else:
		 return get_last(node.children[0])

#returns the children with the least deadline - computation_time 
def get_min(node):
	minimum = float("inf")
	found = False
	for child in node.children:
		if child.d == None:
			found = True
	if found == False:
		#print "setting: ",child.type,"@",child.start_line
		for child in node.children:
			min_tmp = child.d - float(child.in_time)
			if min_tmp < minimum:
				minimum = min_tmp
		return minimum


#sets the deadline for each task
def chetto_deadlines(node):
	if node.parent :
		for p in node.parent:
			p.d = get_min(p)
		for p in node.parent:
			chetto_deadlines(p)

#applys the chetto algorithm to obtain the deadline and arrival time for each task
def chetto(flow_graph, deadline, optimal_flow):
	node = get_last(flow_graph)
	node.d = deadline
	chetto_deadlines(node)
	flow_graph.arrival = 0
	chetto_arrival(flow_graph, optimal_flow)

#gets the cost of the worst flow
def get_cost(flow_list):
	if len(flow_list) == 0:
		return float("inf")
	else:
		return max([flow.time for flow in flow_list])

def chetto_arrival(node, optimal_flow):
	if node.children :
		for child in node.children:
			if child.arrival == None and all_set(child) == True:
				(a, d) = get_max(child, optimal_flow)
				child.arrival = max(a, d)
			chetto_arrival(child, optimal_flow)


def get_max(node, optimal_flow):
	maximum_a = 0
	maximum_d = 0
	for p in node.parent:
		if p.arrival > maximum_a and p.id == node.id:
			maximum_a = p.arrival
		if p.d > maximum_d and p.id != node.id:
			maximum_d = p.d
	return (maximum_a, maximum_d)

#checks if all the parent nodes have the arrival times set
def all_set(node):
	found = True
	for p in node.parent:
		if p.arrival == None:
			found = False
	return found

def get_id(node, optimal_flow):
	for flow in optimal_flow:
		for task in flow.tasks:
			if node.type == task.type:
				return flow.id

def print_schedule(node):
	if node.color == 'white':
		node.color = 'black'
		print node.type," @ ", node.start_line
		print "\t start: ", node.arrival
		print "\t deadline: ", node.d
		print "\t flow: ", node.id
	for n in node.children:
		print_schedule(n)

def create_schedule(graph, num_cores):
	mapped = []
	schedule = ET.Element('Schedule')
	cores = ET.SubElement(schedule, 'Cores')
	cores.text = str(num_cores)
	make_white(graph)
	task_list = generate_task(graph)
	tree = ET.ElementTree(schedule)
	for task in task_list:
		if 'splitted' in task.type:
			serialize_splitted(task, schedule, mapped)
		elif 'BARRIER' not in task.type:
			pragma = ET.SubElement(schedule, 'Pragma')
			id = ET.SubElement(pragma, 'id')
			id.text = str(task.start_line)
			caller_id = ET.SubElement(pragma, 'Caller_id')
			if(len(task.parent) > 0):
				caller_id.text = str(task.parent[0].start_line)
			else:
				caller_id.text = str(0)
			pragma_type = ET.SubElement(pragma, 'Type')
			pragma_type.text = str(task.type)
			threads = ET.SubElement(pragma, 'Threads')
			thread = ET.SubElement(threads, 'Thread')
			thread.text = str(task.id)
			start = ET.SubElement(pragma, 'Start_time')
			start.text = str(task.arrival)
			end = ET.SubElement(pragma, 'Deadline')
			end.text = str(task.d)
			created = False
			if 'BARRIER' not in task.children[0].type :
				l = []
				if 'Parallel' in task.type:
					barrier = ET.SubElement(pragma, 'Barrier')
					created = True
					first = ET.SubElement(barrier, 'id')
					first.text = str(task.start_line)
				if not ('OMPParallelForDirective' in task.type and 'Parallel' in task.children[0].type) and not isinstance(task.children[0], par.Fx_Node):
					if created == False:
						barrier = ET.SubElement(pragma, 'Barrier')
						created = True
					for c in task.children:
						if c.start_line not in l:
							tmp_id = ET.SubElement(barrier, 'id')
							tmp_id.text = str(c.start_line)
							l.append(c.start_line)
			elif ('OMPParallelForDirective' in task.type and 'BARRIER' in task.children[0].type):
				if created == False:
						barrier = ET.SubElement(pragma, 'Barrier')
						created = True
				first = ET.SubElement(barrier, 'id')
				first.text = str(task.start_line)
	par.indent(tree.getroot())			
	tree.write('schedule.xml')

def serialize_splitted(task, schedule, mapped):
	if task.start_line not in mapped:
		pragma = ET.SubElement(schedule, 'Pragma')
		id = ET.SubElement(pragma, 'id')
		id.text = str(task.start_line)
		caller_id = ET.SubElement(pragma, 'Caller_id')
		if(len(task.parent) > 0):
			caller_id.text = str(task.parent[0].start_line)
		else:
			caller_id.text = str(0)
		pragma_type = ET.SubElement(pragma, 'Type')
		pragma_type.text = str(task.from_type)
		threads = ET.SubElement(pragma, 'Threads')
		thread = ET.SubElement(threads, 'Thread')
		thread.text = str(task.id)
		start = ET.SubElement(pragma, 'Start_time')
		start.text = str(task.arrival)
		end = ET.SubElement(pragma, 'Deadline')
		end.text = str(task.d)
		mapped.append(task.start_line)
		if 'BARRIER' not in task.children[0].type :
			l = []
			barrier = ET.SubElement(pragma, 'Barrier')
			if 'Parallel' in task.from_type:
				first = ET.SubElement(barrier, 'id')
				first.text = str(task.start_line)
			if not ('OMPParallelForDirective' in task.from_type and 'Parallel' in task.children[0].type):
				for c in task.children:
					if c.start_line not in l:
						tmp_id = ET.SubElement(barrier, 'id')
						tmp_id.text = str(c.start_line)
						l.append(c.start_line)
		elif ('OMPParallelForDirective' in task.from_type and 'BARRIER' in task.children[0].type):
			barrier = ET.SubElement(pragma, 'Barrier')
			first = ET.SubElement(barrier, 'id')
			first.text = str(task.start_line)
	else:
		for p in schedule.findall("Pragma"):
			if p.find('id').text == task.start_line:
				threads_ = p.find('Threads')
				thread = ET.SubElement(threads_, 'Thread')
				thread.text = str(task.id)

def check_schedule(main_flow):
	make_white(main_flow)
	gen = generate_task(main_flow)
	for node in gen:
		if node.d < 0:
			return False
	return True
\end{lstlisting}

\end{document}