\documentclass[a4paper,11pt,oneside]{book}
\usepackage{latexsym}
\usepackage{braket}
\usepackage{amsfonts}
\usepackage{amsmath}
\usepackage{verbatim}
\usepackage{amssymb}
\usepackage{amsthm}
\usepackage{graphicx}
\usepackage[english]{babel}
\usepackage[utf8]{inputenc}

%%%%%%%%%%%%%
\usepackage{color}
\usepackage[colorlinks]{hyperref}
\hypersetup{urlcolor=darkblue}


\usepackage{listings}
\usepackage{color}
\definecolor{darkblue}{rgb}{0.0,0.0,0.8}
\definecolor{green}{rgb}{0,0.6,0}
\definecolor{gray}{rgb}{0.5,0.5,0.5}
\definecolor{darkgray}{rgb}{0.105,0.105,0.105}
\definecolor{mauve}{rgb}{0.58,0,0.82}
\definecolor{black}{rgb}{0,0,0}
\definecolor{purple}{rgb}{0.72,0.61,0.139}
%\definecolor{ocra}{rgb}{0.204, 0.119, 0.34}
\definecolor{ocra}{rgb}{0.184, 0.134, 0.11}

\lstdefinelanguage{CCC}{ 
  breakatwhitespace=false,         
  breaklines=true,                 
  captionpos=b,                   
  frame=single,                    
  keepspaces=true,                 
  numbers=left,                    
  numbersep=5pt,                   
  numberstyle=\tiny\color{gray}, 
  rulecolor=\color{black},         
  showtabs=false,                  
  stepnumber=1,                    
  tabsize=2,                       
  title=\lstname,                   
 language=C++,
 basicstyle=\ttfamily,
 morekeywords={A}, 
 keywordstyle=\color{darkblue}\ttfamily,
 %identifierstyle=\color{green},
 stringstyle=\color{red}\ttfamily,
 %numberstyle=\color{purple}
 commentstyle=\color{gray}\ttfamily,
 morecomment=[l][\color{mauve}]{\#}
}

\lstdefinelanguage{AST}{ 
  breakatwhitespace=false,         
  breaklines=true,                 
  captionpos=b,                   
  frame=single,                   
  keepspaces=true,                 
  rulecolor=\color{black},         
  showtabs=false,                  
  tabsize=2,                       
  title=\lstname,                   
  morekeywords={A}, 
  keywordstyle=\color{black},
 language=C++,
  %identifierstyle=\color{green},
  stringstyle=\color{ocra},
  %numberstyle=\color{purple},
 basicstyle=\color{gray}\scriptsize
}


\renewcommand{\lstlistingname}{Code}

%%%%%%%%%%%%%%


\pagestyle{plain}


\author{Giacomo Dabisias, Filippo Brizzi}
\title{A framework for static allocation of parallel OpenMp code on multi-core platforms}
\frenchspacing

\begin{document}
\frontmatter
\tableofcontents

\chapter{Abstract}
\section{Objective}
\section{Chapter's structure}


\mainmatter
\chapter{Introduction}
\section{Motivation, context and target application}
\section{Supporting parallelism in C/C++}

In the last years the diffusion of multi-core platforms has rapidly increased enhancing the need of tools and libraries to write multi-threaded programs. 

Several major software companies developed their own multi-thread libraries like Intel with Thread Building Block (TBB) and Microsoft with Parallel Patterns Library (PPL). There are also several open-source libraries like Boost and OpenMP. With the release of C++11 standard also the standard C++ library supports threading.

Lately appeared also automatic parallelization tools. These softwares allow to automatically transform a sequential code into an equivalent parallel one, some of the best examples are YUCCA and the Intel C++ compiler. 

Lastly it is worth to mention briefly GPU, that have become accessible to programmers with the release of tools like Nvidia's CUDA or the open-source OpenCL. These libraries allow the user to easily run code on the GPU; of course the program to run must have some particular features because GPU are still not general purpose. 

In this thesis there was the necessity to find two multi-thread libraries, one used by the input program to describe its parallel sections and the other to run the input program with the static custom schedule. 

The first library had to satisfy the following requirements. It has to allow to transform easily a given sequential real\-time code into a parallel one, without upsetting too much the original structure of the code; given that these kind of code have been usually deeply tested and must meet strict requirements.

What stated above is important also because part of the thesis is to extract informations from the parallel code, such as to distinguish between two parallels regions of code and understand precedences and dependencies between blocks of code. These informations are vital to be able to transform the code in a set of tasks and to provide to the scheduler algorithm the most accurate parameters. Furthermore the parallel code has to be instrumented both to profile it and to divide it into tasks for the final execution. The analysis of the various frameworks didn’t involved their execution performance as the aim of the thesis is to use the library calls just to create a database containing the information regarding the parallel structure of the code.  

For these reasons OpenMP has soon been considered the best choice. OpenMP behaviour is described in the chapter~\ref{chap:openmp}, here will be roughly presented some of its features, that are necessary to understand the reason of the choice.  

First of all OpenMP is minimally invasive as it just adds annotations inside the original sequential code, without any needs of changing its structure. OpenMP works embedding pieces of code inside scopes and adding annotations to each of these scopes by means of the pragma construct. The scopes automatically identify the distinct blocks of code (tasks) and also give immediately some information about the dependencies among the differents blocks. The use of pragmas is very convenient as they are skipped by the compiler if not informed with a specific flag. This implies that, given an OpenMP code, to run it sequentially is just enough to not inform the compiler of the presence of the OpenMP. This feature will be usefull to profile the code. Finally OpenMP is well supported by the main compilers and it has a strong and large community developing it, including big companies such as Intel.  

The second library had instead opposite requirements as it has to be highly efficient. For this reason has been chosen the C++ standard library. Since the release of the C++11 standard, the C++ standard library was provided with threads, mutex, semaphore and condition variables. This library has the drawback of being not easy to use when coming to complicated and structured tasks, but on the other side it is fully customizable as it allows to instantiate and use directly system threads. It provides the chance to directly tuning the performance of the overall program and differently from the other parallelization tools the \emph{std} library allows to allocate each task on a specific thread.




\section{The OpenMP standard}
\label{chap:openmp}

\section{Clang as LLVM frontend}

Clang \cite{clang} is a compiler front-end for the C, C++ and Objective-C programming languages. It relies on LLVM as its back-end. 

A compiler front-end is in charge of analyzing the source code to build the intermediate representation (IR) which is an internal representation of the program. The front\-end is usually implemented as three phases: lexing, parsing, and semantic analysis. This helps to improve modularity and separation of concern and allows programmers to use the front\-end as a library in their projects. 

The IR is used by the compiler back\-end (LLVM in the case of Clang), that transforms it into machine language, operating in three macro phases: analysis, optimization and code generation. 

The Clang project was started by Apple and was open-sourced in 2007. Nowadays its development is completely open-source and besides Apple there are several major software companies involved, such as Google and Intel.

Clang is designed to be highly compatible with GCC, Its command line interface is similar to and shares many flags and options with GCC. Clang was chosen for the development of the thesis over GCC for three main reasons. Clang has proven to be faster and less memory consuming in many situations \cite{clanggcc}. Clang has a modular, library based architecture. This structure allows the programmer to easily embed Clang’s functionalities inside its own code. Each of the libraries that forms Clang has its specific role and set of functions; in this way the programmer can simply use just the libraries he needs, without having to study the whole system. On the other side GCC design makes difficult to decouple the front-end from the rest of the compiler.
The third and most important Clang’s feature is that it provides the possibility to perform code analysis, extract informations from the code and, most important, to perform source-to-source transformation.

Clang was not the only possibility out of GCC, also the Rose Compiler and Mercurium were viable options. 

The strength of Clang, is in its implementation of the Abstract Syntax Tree (AST). Clang’s AST is different from ASTs produced by some other compilers in that it closely resembles both the written C++ code and the C++ standard.

The AST is accessed through the \emph{ASTContext} class. This class contains a reference to the \emph{TranslationUnitDecl} class which is the entry point into the AST (the root). It also provides the methods to traverse the AST.

Clang’s AST nodes are modeled on a class hierarchy that does not have a common ancestor. Instead, there are multiple larger hierarchies for basic node types. Many of these hierarchies have several layers and bifurcations so that the whole AST is composed by hundreds of classes for a total of more than one hundred thousand lines of code. Basic types derive mainly from three main disjoint classes: \emph{Decl}, \emph{Type} and \emph{Stmt}. 

As the name suggests the classes that derive from the \emph{Decl} type represent all the nodes matching piece of code containing declaration of variables (\emph{ValueDecl}, \emph{NamedDecl}, \emph{VarDecl}), functions (\emph{FunctionDecl}), classes (\emph{CXXRecordDec}l) and also function definitions.
  
The Clang’s AST is fully type resolved and this is afforded using the \emph{Type} class which allows to describe all possible types (\emph{PointerType}, \emph{ArrayType}).

Lastly there is the \emph{Stmt} type which refereres to control flow (\emph{IfStmt}) and loop block of code (\emph{ForStmt}, \emph{WhileStmt}), expressions (\emph{Expr}), return command (\emph{ReturnStmt}), scopes (\emph{CompoundStmt}), etc.. 

Together with the above three types there are other “glue” classes that allow to complete the semantic. The most remarkable ones are: the \emph{TemplateArgument} class, that, as the name suggests, allows to handle the template semantic and the \emph{DeclContex} class that is used to extend \emph{Decl} semantic and that will be shown later. 

To built the tree the nodes are connected to each other; in particular a node has references to its children. For example a \emph{ForStmt} would have a pointer to the \emph{CompoundStmt} containing its body, as well to the \emph{Expr} containing the condition and the \emph{Stmt} containing the initialization. Special case is the \emph{Decl} class that is designed not to have children thus can only be a leaf in the AST. There are cases in which a \emph{Decl} node is needed to have children, like for example a \emph{FunctionDecl}, which has to refer to the \emph{CompoundStmt} node containing its body or to the list of its parameters (\emph{ParmVarDecl}). The \emph{DeclContext} class has been designed to solve this issue. When a \emph{Decl} node needs to have children it can just extend the \emph{DeclContext} class and it will be provided with the rights to points to other nodes. 

There are other two classes that are worth speaking about: \emph{SourceLocation} and \emph{SourceManager} class. The \emph{SourceLocation} class allows to map a node to the source code. The SourceManager instead provides the methods to calculate the location of each node. These classes are very powerful as they allow to retrieve both the start and the end position of a node in the code, giving the exact line and column number. For example given a \emph{ForStmt}, the \emph{SourceManager} is able to provide the line number of where the stmt starts and ends, but also the column number where the loop variable is declared or where the increment is defined.

To traverse the AST the Clang provides the \emph{RecursiveASTVisitor} class. This is a very powerful and quite easy to learn interface that allows the programmer to visit all the AST’s nodes. The user can customize this interface in such a way it will trigger only on nodes he is interested about; for example the methods \emph{VisitStmt()} or \emph{VisitFunctionDecl()} are called each time a node of that type is encountered. Each AST’s node class contains getter methods to extract informations out of the code. For example a \emph{Stmt} class has a method to know what kind of \emph{Stmt} is the node, as \emph{IfStmt}, \emph{Expr}, \emph{ForStmt}, etc.. In turn \emph{ForStmt} class provides methods to find out the name of the loop variable, it’s initial value and the loop condition.

To better understand how the Clang’AST is structured, Code~\ref{astexemple} and~\ref{ast} contain a simple code and the associated AST.
\newline
\begin{lstlisting}[language=CCC, caption=Simple code., label=astexemple]
class A {
public:
	int x;
	void set_x(int val) {
		x = val * 2;
	}	
	int get_x() {
		return x;
	}
};
int main() {
	A a;
	int  val = 5;
	a.set_x(val);
}
\end{lstlisting}

\begin{lstlisting}[language=AST, caption=Clang AST of the simple code., label=ast]
TranslationUnitDecl
|-CXXRecordDecl <clang_ast_test.cpp:2:1, line:13:1> class A
| |-CXXRecordDecl <line:2:1, col:7> class A
| |-AccessSpecDecl <line:3:1, col:7> public
| |-FieldDecl <line:4:2, col:6> x 'int'
| |-CXXMethodDecl <line:5:2, line:7:2> set_x 'void (int)'
| | |-ParmVarDecl <line:5:13, col:17> val 'int'
| | `-CompoundStmt <col:22, line:7:2>
| |   `-BinaryOperator <line:6:3, col:13> 'int' lvalue '='
| |     |-MemberExpr <col:3> 'int' lvalue ->x
| |     | `-CXXThisExpr <col:3> 'class A *' this
| |     `-BinaryOperator <col:7, col:13> 'int' '*'
| |       |-ImplicitCastExpr <col:7> 'int' <LValueToRValue>
| |       | `-DeclRefExpr <col:7> 'int' lvalue ParmVar 'val' 'int'
| |       `-IntegerLiteral <col:13> 'int' 2
| |-CXXMethodDecl <line:9:2, line:11:2> get_x 'int (void)'
| | `-CompoundStmt <line:9:14, line:11:2>
| |   `-ReturnStmt <line:10:3, col:10>
| |     `-ImplicitCastExpr <col:10> 'int' <LValueToRValue>
| |       `-MemberExpr <col:10> 'int' lvalue ->x
| |         `-CXXThisExpr <col:10> 'class A *' this
| |-CXXConstructorDecl <line:2:7> A 'void (void)' inline
| | `-CompoundStmt <col:7>
| `-CXXConstructorDecl <col:7> A 'void (const class A &)' inline
|   `-ParmVarDecl <col:7> 'const class A &'
`-FunctionDecl <line:15:1, line:21:1> main 'int (void)'
  `-CompoundStmt <line:15:12, line:21:1>
    |-DeclStmt <line:17:2, col:5>
    | `-VarDecl <col:2, col:4> a 'class A'
    |   `-CXXConstructExpr <col:4> 'class A' 'void (void)'
    |-DeclStmt <line:18:2, col:14>
    | `-VarDecl <col:2, col:13> val 'int'
    |   `-IntegerLiteral <col:13> 'int' 5
    `-CXXMemberCallExpr <line:20:2, col:13> 'void'
      |-MemberExpr <col:2, col:4> '<bound member function type>' .set_x
      | `-DeclRefExpr <col:2> 'class A' lvalue Var 'a' 'class A'
      `-ImplicitCastExpr <col:10> 'int' <LValueToRValue>
        `-DeclRefExpr <col:10> 'int' lvalue Var 'val' 'int'
\end{lstlisting}


Clang supports the insertion of custom code inside the input one through the \emph{Rewriter} class. This class provides several methods that allow, specifying a \emph{SourceLocation}, to insert, delete and replace text; it also allows to replace a \emph{Stmt} object with another one. The programmer cannot know a priori the structure of the input source code, so the best way to insert the custom text, in the correct position, is during the parsing of the AST. It is in fact possible to access each node’s start and end \emph{SourceLocations} reference, to transform them in a line plus column number and insert the text at the end of the line or at line above or below, as needed by the program. 

The inserted text and its position are stored during the parsing of the AST in a buffer inside the \emph{Rewriter} object; when the parsing is completed the buffer’s data is inserted in the code generating a new file. 

Clang’s support to pragmas and OpenMP is really recent. Intel provided an unofficial patched version of the original Clang, which fully supports the OpenMP 3.3 standard, in July 2013 and the patch has not yet been inserted in the official release. Although it is not an official release Intel has worked inline with the Clang community principle and design strategies and it also produced a good Doxygen documentation of the code. This patch works jointly with the Intel OpenMP Runtime Library \cite{intelomprtl}, which is open-source. 

For what concern the support to generic pragmas the only remarkable work, that goes close to this goal is the one of Simone Pellegrini. He indeed implemented a tool (Clomp \cite{clomp}) to support OpenMP pragmas in Clang. Clomp is implemented in a modular and layered way; this implies that the same structure can be easily used to support customized pragmas.
 


\chapter{Design}
\section{The framework}
\section{A simple example}
\section{Analysis}
\subsection{Code}

In this chapter will be presented and justified the choices that have been made during the parsing of the input source code; in particular which features have been extracted and why.

First of all will be presented how are translated OpenMP pragma in the Clang’s AST. Our framework targets only a small part of the OpenMP environments, in particular only \emph{parallel}, \emph{sections} (\emph{single}), \emph{section} (\emph{task}) and \emph{for} pragmas. These pragmas have the common property that are all transformed into \emph{Stmt} nodes in the AST. Each of these pragmas is represented in the AST by a specific class: \emph{OMPParallelDirecetive}, \emph{OMPSectionsDirective} and so on. All these class inherit from the \emph{OMPExecutableDirective} class which in turn derives from the \emph{Stmt} class. 

These classes has three main functions, one to know the name of the directive associated to it, one to retrieve the list of the clauses associated to the pragma and the last to get the stmt associated to the pragma. Based on this last function the above directives can be divided into two groups, the first containing the \emph{for} pragma and the second all the others. The difference between the two groups is that the \emph{for} pragma has associated a \emph{ForStmt}, while the other have associated a \emph{CompoundStmt}. All the clauses derived from a common primitive ancestor which is the \emph{OMPClause} class. 

A real-time program, to be scheduled, needs to provide some informations about its time constraints, in particular the deadlines; these data can be provided in a separate file or directly inside the code. In this framework has been chosen the second approach and it has been done using the OpenMP clauses. The standard clauses clearly don’t allow to specify the deadline of a pragma, so has been created a patch at the standard Clang to support of the \emph{deadline} clause. This patch can be further enhanced to support other custom clauses, such as the activation time or the period of the pragma.

The thesis’s framework parses the source code customizing the \emph{RecursiveASTVisitor} interface; in particular it overrides two methods: \emph{VisitFunctionDecl()} and \emph{VisitStmt}. Each time the parser comes up with a \emph{FunctionDecl} object or a \emph{Stmt} object it invokes the associated custom function \emph{VisitFunctionDecl()} adds all objects representing a function definition to a \emph{FIFO} list.  At the end of the parsing this list will contain the definition of all the functions in the input source code. \emph{VisitStmt()} instead triggers on each stmt, it checks the type and if it is an \emph{OMPExecutableDirective} node it adds it to another \emph{FIFO} list, that at the end will contain all the pragmas. The two lists have the property that the order of their elements is given by the positions of the nodes in the source code, the smaller the starting line of a node the smaller its position in the list. This property is granted by the fact that the input code is parsed top down.

Once all the pragmas are in the list, the tool inspects each node, extracting information and saving them in a custom class. The newly created objects are used to build a pragmas tree. Since an input code can have multiple functions containing OpenMP pragmas and at static time it is not possible to understand where and when these functions will be called, the framework builds different pragmas tree for each function containing pragmas. It is possible to know, for each function, at which line its body starts and ends and so it is possible to match each pragma to the correct function. The tree structure is given by the nested architecture of the OpenMP pragmas, that has been described in chapter~\ref{openmp}. The built of the tree is quite simple and straightforward as there are several properties that come to handy. The extracted pragmas in the list are ordered according to their starting line, so pragmas belonging to the same function are continuous. Every time a pragma is popped from the list, its starting line is checked, if it belongs to the same function of the previous node it is added to the current tree, otherwise it will be the root of a new tree. Another property is that a pragma is a child of another pragma, only if it is nested inside it; to be nested a node must have its starting line greater and its ending line smaller than the other one.  The last property, that still comes from the ordered pragma list and from the definition of nested, is that a node can be nested only in its previous node (in the list) or in the father of the previous node, or in the father of the father and so on.

During the creation of the trees each AST node is transformed in a custom object that will carry only the information useful for the framework. 

\begin{itemize}
\item Pragma type: parallel, sections, for, etc.
\item Start line and end line of the statement associated with the pragma.
\item A reference to the object containing the information of the function where the pragma is defined.
\item A reference to the original AST node.
\item The list of the pragma’s clauses and of the variables involved. 
\item The list of its children nodes and a reference to its parent node.
\item In case the node is of type \emph{for} or \emph{parallel for} it contains the reference to another object that contains all the information of the for:
\begin{itemize}
\item the name of the loop var, its type and initial value.
\item The name of the condition variable, or the condition value.
\item The increment.
\end{itemize}
\end{itemize}

The framework support only the parsing of for with a simple structure: \newline

$parameter = value \ | \ var$

$c\_op \ = \ < \ | \ > \ | \ <= \ | \ >= $

$i\_op \ = \ ++ \ | \ -- \ | \ += \ | \ -= \ | \ *=$ 

$for([type] \ var \ = \ parameter; \ var \ c\_op \ parameter; \ var \ i\_op \ [parameter])$ \newline


The \emph{ForStmt} class fully supports the C++ For semantic, this means that it would be possible for the framework to support any kind of For declaration. It has been chosen to support only a basic structure because the effort required to expand the semantic it’s very high and, with some slightly modification to the code, it is possible to support almost any possible scenarios. For example a for declaration like this:

\begin{lstlisting}[language=CCC]
for (int i = foo(); i < bar*baz; i ++)
\end{lstlisting}
can be translated as:

\begin{lstlisting}[language=CCC]
int init_val = foo();
int cond_val = bar*baz;
for(int i = init_val; i < cond_val; i ++)  
\end{lstlisting}

becoming understandable by the framework.

Once all the pragmas have been translated and put in a tree, the new data structures are translated in the xml format. Each object is described either by a \emph{Pragma} tag or by a \emph{Function} tag. The two tags contains a list of other tags, one for  each of the variables contained in the tree’s objects. The semantic of the xml allows also to translate perfectly the original tree structure. This is done nesting \emph{Pragma} tags one inside the other. The outermost tags are of type \emph{Function}. Each function is the root of a tree so it will contain one or more \emph{Pragma} tags. In turn each \emph{Pragma} tag, if has children in the original tree, will contain other \emph{Pragma} tags. Code~\ref{xmlpragma} represent e portion of the xml generated from the sample code in section~\ref{exemple}.

\begin{lstlisting}[language=XML, caption=XML file of the pragma structure of Code~\ref{code}., label=xmlpragma]
<File>
    <Name>omp_test.cpp</Name>    
    ...
    <Function>
        <Name>main</Name>
        <ReturnType>int</ReturnType>
        <Parameters>
            <Parameter>
                <Type>int</Type>
                <Name>argc</Name>
            </Parameter>
            <Parameter>
                <Type>char **</Type>
                <Name>argv</Name>
            </Parameter>
        </Parameters>
        <Line>12</Line>
        <Pragmas>
            <Pragma>
                <Name>OMPParallelDirective</Name>
                <Options>
                    <Option>
                        <Name>private</Name>
                        <Parameter>
                            <Type>int</Type>
                            <Var>bar</Var>
                        </Parameter>
                    </Option>
                </Options>
                <Position>
                    <StartLine>15</StartLine>
                    <EndLine>30</EndLine>
                </Position>
                <Children>
                    <Pragmas>
                        <Pragma>
                            <Name>OMPSectionsDirective</Name>
                            <Position>
                                <StartLine>17</StartLine>
                                <EndLine>29</EndLine>
                            </Position>
                            <Children>
                                <Pragmas>
                                    <Pragma>
                                        <Name>OMPSectionDirective</Name>
                                        <Position>
                                            <StartLine>19</StartLine>
                                            <EndLine>22</EndLine>
                                        </Position>
                                    </Pragma>
				...
</File>
\end{lstlisting}
This xml file will then be passed to the scheduler algorithm, that will add a semantic to each node to build a parallelization graph, that will be used to create the tasks’ schedule. The original trees are not discarded and they will be used to produce the final code, during a following parsing of the code.




\subsection{Parallelism}
\section{Intrumentation for profiling}

To produce a good schedule the framework needs informations about the tasks, in particular their computation time, their relations and precedences. In paragraph~\ref{code} we have seen how the pragmas are extracted and their structure; what was left is to retreive the computation time of each task and the functions’ call graph. The only way to get these informations is to profile at run\-time the sequential version of the input code; to have the sequential version, given that the code is parallelized with OpenMP, is enough to compile it without the \emph{\-fopenmp} flag. 

To be profiled the code needs to be instrumented. In the original code are added calls to a run-time support which calculates the time of the tasks, tracks the caller id of each function and stores them in a log file. 

The instrumentation is performed during the first parse of the code, when the pragma statements are collected; the instrumentation does not depend on the semantic of each pragma and makes no distinction between functions and pragma. The idea is that, each time during the parsing a pragma or a call to a function is met, a call to the run\-time support is inserted. As we have seen in paragraph~\ref{code}, both functions and pragmas have associated either a \emph{CompoundStmt} or a \emph{ForStmt} node. A \emph{CompoundStmt} has the characteristic that it always represents a block of code enveloped in a couple of curly brackets (a scope). In C++ variables declared inside a scope are locally to it, so are destroyed when the scope ends; the idea is to insert in the original code, at the beginning of the scope, a call to a custom object constructor, that starts a timer. When the scope ends the inserted object is destroyed and its destructor called; the destructor stops the timer and saves the time in a log file. 

The tasks can be nested to each other, this means that an outermost task computation time contains the computation time of its sub\-tasks; in other words, the sum of all the tasks’ computation time could exceed the total computation time of the program. To obviate at this problem has been designed a method so that each task can keep track of the computation time of its children so that it is possible to obtain its effective computation time. This method allows also to keep track of the caller identity of each pragma or function; the caller is always either another pragma or a function.

This method works as follows: there is a global variable that stores the identity of the current pragma or function in execution. Each time a pragma or a function starts its execution the profiler object is allocate and its constructor invoked. The function puts a reference of the pragma\/function, where it is defined, in the global variable and saves the old value as it identifies its caller. When the object is destroyed the destructor is invoked and it communicates to its caller its computation time, so that the other task can increment the variable containing the children computation time. Before ending the destructor swap again the value of the global variable, passing to it the identifier of its caller. 
In case of a For task the profiler evaluates the number of iterations; this is very important because it helps the scheduler algorithm to decide how much to split the For in the final parallel execution. This evaluation is done subtracting the initial value from the ending value and dividing for the increment. This method is not perfect because it may happen that the values of the loop variable or of the conditional variable are changed inside the For body, changing the number of iterations; however the framework’s target applications are real-time programs, so it is very unlikely to find dynamic For block. A possible solution to this problem would be to create a new variable, initialized to zero, that it is incremented by one at each iteration and when the For completes its value is caught and stored in the log file. At the end the log file will contain for each task:

\begin{itemize}
\item The total time of the task, from when it was activated since it terminates.
\item The time of all its nested tasks, to calculate the effective time just perform the difference with the total time.
\item The identifier of the pragma or function that called the task.
\item In case of For task the number of iteration.
\end{itemize}

Code~\ref{xmllob} shows the log file of the code~\ref{code}.

\begin{lstlisting}[language=XML, caption=XML file of the pragma structure of Code~\ref{code}., label=xmlpragma]
<LogFile>
	<Hardware NumberofCores="4" MemorySize="2000"/>
	<Pragma fid="3" pid="5" callerid="3" elapsedTime="6" childrenTime="0" loops="6"/>
	<Function fid="3" callerid="19" elapsedTime="6" childrenTime="6"/>
	<Pragma fid="12" pid="19" callerid="17" elapsedTime="6" childrenTime="6"/>
	<Pragma fid="3" pid="5" callerid="3" elapsedTime="8" childrenTime="0" loops="8"/>
	<Function fid="3" callerid="25" elapsedTime="8" childrenTime="8"/>
	<Pragma fid="12" pid="25" callerid="17" elapsedTime="8" childrenTime="8"/>
	<Pragma fid="12" pid="17" callerid="15" elapsedTime="14" childrenTime="14"/>
	<Pragma fid="12" pid="15" callerid="12" elapsedTime="14" childrenTime="14"/>
	<Function fid="12" elapsedTime="14" childrenTime="14"/>
</LogFile>
\end{lstlisting}


\section{Profiling}
\section{Schedule generation}
\section{Instrumentation for the execution}

In this paragraph will be presented the design strategies that have been used to instrument the input code to make it run accordingly to the schedule produced by the framework. 

The framework needs to be able to isolate each task and execute it in the thread specified by schedule; to do so new lines of code are added in the original code to transform the old pragma in a collection of real atomic independent tasks. In this phase the functions are not considered as tasks and they won’t be affected by the instrumentation. This is due to the fact that functions have no parallel semantic themselves and they can be simply executed by the tasks that invoke them, without affecting the semantic and improving the efficiency.

The idea of this phase is to transform each pragma block of code into a function, that will be called by the designated thread. One possibility was to take the code of the pragma, remove it from the function where it is defined and put it in a newly generated function; this way may be feasible with the Clang’s tools but it is very complicated because of the presence of nested pragmas. 

The other possibility, the one used in the framework, is to exploit, once again, the property of the pragmas to be associated with a scope. In C++ it is possible to define a class inside a function if the class is contained in a scope. Exploiting this property each pragma code has been enveloped inside a class declaration; in particular it constitutes the body of a function defined inside the new class.

In the case of a \emph{for} pragma the framework needs to perform some additional modifications to the source code. Usually a For is splitted on more threads in the final execution so the For declaration has to be changed to allow the iterations to be scattered between different threads. In the For declaration are added two variables: an identifier, to distinguish the different threads and the number of threads concurring in the execution of the For. Here an example:

\begin{lstlisting}[language=CCC]
for(int i = begin; i < end; i ++)
\end{lstlisting}

becomes

\begin{lstlisting}[language=CCC]
int id; //incremental identifier of the task
int num_threads; // number of threads concurring in the execution of the for;
for(int i = begin + id * (end - begin) / num_threads; i < (id + 1) * (end - begin) / num_threads; i ++)
\end{lstlisting}

so if $num\_threads  = 4$, $begin = 0$, $end = 16$, each thread will execute four iterations, in particular, the third thread, with $id = 2$ (identifier starts always from zero) will execute

\begin{lstlisting}[language=CCC]
int new_begin = 0 + 2 *(16 - 0) / 4;
int new_end = 0 + (2 + 1) * (16 - 0) / 4;
for(int i = 8; i < 12; i ++)
\end{lstlisting}

After the definition of the class, at the end of the scope, the framework adds a piece of code that instantiates an object of the created class and pass it to the run-time support. The object will be collected by the designated thread which will invoke the custom function that contains the original code, running it. 

This approach does not change the structure of the code, in particular nested pragmas remain nested; this means that there will be classes definition inside others classes, more precisely there will be tasks inside other tasks. This may seems a problem because it creates dependencies between tasks, not allowing a fully customizable schedule, but this is not true. According to the OpenMP semantics each task is not fully independent to the others, there can be precedences in the execution, but this approach grants that if two tasks can be run in parallel there will be no dependencies between them. To understand this we have to remind the OpenMP structure illustrated in paragraph~\ref{openmp}, where it is explained that two pragmas containing computation code  can be related only if in two different functions and so they won't be nested in the source code. 




\section{Run-time support}

In this chapter will be presented how has been designed the run-time support for the execution of the final program. The aim of the run-time is to instantiate and manage the threads and to control the execution of the tasks. In particular it must allocate each task in the correct threads and must grant the precedence constraints between tasks setting the semaphores order. The run\-time must be very fast to grant that the time constraints of the tasks are always satisfied. For this reason the run\-time does no time consuming computations and all its allocation decisions are made based on what is written in the schedule. All the complicated calculation to decide the tasks allocation has been already done by the scheduler algorithm before the program execution and the produced schedule is taken as input by the program. 

Now will be show step by step the execution of the run\-time. First of all it parses the schedule file extracting all the informations and storing them in its own variables. It then instantiates a threads pool as large as specified in the schedule and it creates a job queue for each of thread. 

The main program invokes the run-time support passing to it the object containing the function to be executed. The run-time embeds the received object in ad\-hoc class, that includes the methods and variables needed to perform synchronization on that task. The created job is inserted in a vector common to all threads; at this point the run\-time searches in the schedule to find which thread has been designated to run that job and puts an identifier of the job in that thread’s job queue. In case of a For task the run\-time has to execute some additional steps. Usually a For task is splitted on more threads, so the run-time has to duplicate the task for each thread involved; each copy is initialized with an incremental identifier, starting from zero and it also receives the total number of threads concurring at the execution of the task. These values are mandatory to inform each thread which iterations of the For has to execute. 

Each thread executes an infinite loop. At the beginning of each iteration the thread checks if its queue contains a references to a job, in that case it pulls the first identifier and uses it to retrieve the real job in the vector of jobs and executes it. When the job ends the thread checks the schedule to see if it has to wait for other tasks to complete. Then the thread notifies that the job it was executing has been completed, so that any other thread waiting on that job can continue its executions. The common jobs’ vector is needed because it allows to share information of a task between all the threads, in particular it is mandatory to perform task synchronization. In Code~\ref{code} the \emph{Sections} task at line 17, once has launched its children tasks, has to wait for them to complete in order to finish. This rule is true for each “control” pragma that has children. 


 


\chapter{Implementation}
\section{Scheduling XML schema}
\section{Instrumentation for Profiling}
\section{Profiling implementation}
\section{Schedule generating tool}
\section{Instrumentation for the execution}
\section{Run-time support}

\chapter{Performance evaluation}
\section{A computer vision application}
\section{Results with statistics}

\chapter{Conclusions}
\section{Achieved results}
\section{Future development}






\begin{thebibliography}{9}
\bibitem{bbw} Giorgio Buttazzo, Enrico Bini, Yifan Wu. \emph{Partitioning parallel applications on multiprocessor reservations}. Scuola Superiore Sant’Anna, Pisa, Italy
\bibitem{bb} Giorgio Buttazzo, Enrico Bini, Yifan Wu. \emph{Partitioning real-time applications over multi-core reservations}. Scuola Superiore Sant’Anna, Pisa, Italy
\bibitem{mfp} Ricardo Garibay-Martinez, Luis Lino Ferreira and Luis Miguel Pinho, \emph{A Framework for the Development of Parallel and Distributed Real-Time Embedded Systems}
\bibitem{pop}Antoniu Pop (1998). \emph{OpenMP and Work-Streaming Compilation in GCC}. 3 April 2011, Chamonix, France

\bibitem{clang}\url{http://clang.llvm.org/}
\bibitem{clanggcc}\url{http://clang.llvm.org/features.html\#performance}
\bibitem{intelomprtl}\url{https://www.openmprtl.org/}
\bibitem{clomp}\url{https://github.com/motonacciu/clomp/}

\end{thebibliography}

\end{document}